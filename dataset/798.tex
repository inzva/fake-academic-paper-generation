\documentclass{bmvc2k}

\usepackage{amsmath} % assumes amsmath package installed
\usepackage{amssymb}  % assumes amsmath package installed
\usepackage{array,multirow}


\title{Deep Representation Learning for Domain Adaptation of
Semantic Image Segmentation}

\addauthor{Assia Benbihi}{abenbihi@georgiatech-metz.fr}{1}
\addauthor{Matthieu Geist}{matthieu.geist@univ-lorrainefr}{2}
\addauthor{Cedric Pradalier}{cedric.pradalier@georgiatech-metz.fr}{1}

\addinstitution{
 UMI 2958 GT-CNRS -- 
 GeorgiaTech Lorraine\\
 Metz, France
}
\addinstitution{
 Universit\'e de Lorraine\\
 CNRS LIEC UNR 7360,\\
 Metz, France
}

\runninghead{Benbihi, Geist, Pradalier}{Deep Representation Learning}

\def\eg{\emph{e.g}\bmvaOneDot}
\def\Eg{\emph{E.g}\bmvaOneDot}
\def\etal{\emph{et al}\bmvaOneDot}

\begin{document}

\maketitle

\begin{abstract}
Deep Convolutional Neural Networks have pushed the state-of-the art for semantic segmentation provided
that a large amount of images together with pixel-wise annotations is available. Data collection is 
expensive and a solution to alleviate it is to use transfer learning. 
This reduces the amount of annotated data required for the network training
but it does not get rid of this heavy processing step. 
We propose a method of transfer learning without annotations on the target task for
datasets with redundant content and distinct pixel distributions. 
Our method takes advantage of the approximate content alignment of the images 
between two datasets when the approximation error prevents the reuse of annotation from one dataset 
to another. Given the annotations for only one dataset, we train a first network in 
a supervised manner. This network autonomously learns to generate deep data representations
relevant to the semantic segmentation. Then the images in the new dataset, 
we train a new network to generate a deep data representation that matches 
the one from the first network on the previous dataset. 
The training consists in a regression between feature maps and does not require any annotations on the 
new dataset. We show that this method reaches performances similar to a classic transfer learning
on the PASCAL VOC dataset with synthetic transformations.
\end{abstract}

\section{Introduction}
\label{sec:intro}

\begin{figure}[thb]
  \centering
  \includegraphics[width=\linewidth]{training_pipeline.PNG}
  \caption{Top: The trained and frozen (gray) network provides ground truth deep representations. Down: The trainable layers (blue) must learn the deep representations.}
  \label{fig:method}
\end{figure}


Many applications such as environment monitoring or medical image processing 
rely of semantic segmentation of datasets with redundant content. 
Deep Convolutional Neural Networks (DCNN) reach state-of-the-art performance
on segmentation on such datasets but require a large amount of data 
together with pixel-wise annotations. 
These types of applications share a common feature which is that the segmentation dataset gets larger with time and the data distribution may change. This prevents a network 
trained on a dataset at time t to generalize a latter time. 
Each time the pixel distribution deviates too much from the past, 
the network needs to be fine-tuned or retrained from scratch. In
both cases, one needs to provide new pixel-wise annotations as the dataset changes. 
To avoid the burden of reannotation, we propose a method to transfer the learning from a network trained at
time t to the dataset at time t+1 without annotations on the new dataset. To do
so, we transfer the deep representations learned by the network at time t 
instead of transferring the whole segmentation learning. 
This method has been studied in \cite{bengio2012deep} in the Unsupervised and 
Transfer Learning Challenge for classification. 
One of the limitation in this work was the lack of
instances pairs with the same semantic content. In the applications we address, 
such pairs arise naturally which solves this issue. 

For example,
environment monitoring surveys the same area across time. This generates several
datasets with approximately aligned images across surveys. 
One could provide annotations for a
small part of the dataset to train the network and then generalize it to the future.
However,
the image distribution may change across time which prevent a good generalization.
In \cite{richard2018}, a DCNN is used to segment land categories on aerial images over a 
period from 1955 to 2015. The 2015 dataset is made of RGB digital images whereas 
the 1955 data is made of black and white analog images that have been digitized.
A network trained on the 2015 data can not generalize on the 1955 because
the images do not have the same color domain nor the same resolution or grain. 
Also, even though 
the surveys cover the same area, the images are not perfectly aligned due to small changes in land use, which prevent
the use of 2015 annotations to train a network on the 1955 data. 
What stays approximately invariant for one image across surveys is the 
high-level semantic content. DCNN have the property to autonomously build 
high-level data representation in the feature maps during supervised training. 
Our method proposes to train a network $H_1$ only on one of the dataset 
$D_1$ with 
pixel-wise supervision. Then a second network $H_2$ is trained 
to generate deep representations for another dataset $D_2$
so as to match the deep representation of $H_1$ on $D_1$. 

We apply this method to synthetic transformations on the PASCAL VOC 12. We use 
synthetic data to gather ideal baselines that require to have pixel-wise 
annotations on 
all datasets. We show that our transfer learning method reach the same
performance as classic transfer learning and classic training. 
The next section describes the state-of-the-art in transfer learning for DCNN. 
Section 3 and 4 presents our method and results.

\section{Related work}

Transfer learning \cite{pan2010survey} is relevant for DCNN because convolution filters
can be transferred as-is from one network to another and it solves a main issue 
which is collecting enough data for the network training to converge. 
There are two main categories of transfer learning that the literature addresses: 
inductive transfer learning and transductive transfer learning. Inductive transfer 
learning consists in transferring the weights of a network trained on a different 
task but on the same dataset distribution.
This is relevant because the early layers of a DCNN
capture low-level image features such as colours or edges 
\cite{zeiler2014visualizing} which are 
agnostic to the task. Given one dataset, they can be reused as-is as a
foundation on which to 
build high-level features that will be specialised to the task at hands in deeper layers. 
For example, \cite{long2015fully} extend a classification network with fully 
connected layer to a fully convolutional layer for semantic segmentation. 
They keep the first layers of the network which build high level
representations
such as object contours and replace the fully connected layers with convolutional
layers. These last layers are trained from scratch and compute the segmentation
labels from the representation built by the lower layers. State-of-the art segmentation networks
 such as DeepLab \cite{chen2018deeplab} and SegNet \cite{badrinarayanan2017segnet}
 also use this approach.

Transductive transfer on the other hand consists in transferring the weights of a network trained 
to do the same task but on an other dataset. This is a classic method to decrease
the amount of data needed to train a network. For example, \cite{ilg2017flownet} 
propose
to train a network to compute the flow between two images with supervised learning. However, it is
extremely complex to generate real data with ground truth flow. So the authors created the Flying Chair
synthetic dataset to train the network. Then, they trained this network on a reduced 
set of the target dataset. Pre-training on synthetic data is also relevant for 
urban scenes segmentation as demonstrated in~\cite{ros2016synthia}. Our method is similar in that we initialise 
$H_2$ with the weights of the supervised training of $H_1$. However, we do not 
train $H_2$ on the target task which is semantic segmentation. We train $H_2$ 
to generate the same high level representation on $D_2$ as $H_1$ does on $D_1$ 
even though the style of the datasets is different. We do so by taking advantage 
of the content similarity between $D_1$ and $D_2$ which allows us to bypass 
the need for $D_2$ annotations.   

\cite{bengio2012deep} addresses the issue of learning a ``good representation'' in the 
context of The Unsupervised and Transfer Learning Challenge. 
The test sets have examples which are not well represented in the training set.
As a result, the input distribution of the training set is very different what appears in the test set and few or no labels are provided for some of the classes
of interest. This is similar to the problem we address where $D_2$ has a different
distribution from $D_1$ and there are no annotations for $D_2$. The solution they propose
is to learn relevant representations for the non-annotated data using 
an unsupervised representation-learning algorithm. Their results 
are not as satisfying as they expect them to be and their explanation is that they
lack annotated images. They argue that some of the relevant features should be 
learned on other instances than the non-annotated ones. So they use the annotated
training set to build a set of relevant features for the training set and select
the one that were also relevant for the non-annotated test set. 
We follow their guidelines as we first learn relevant semantic representations 
of the image content during the supervised training of $D_1$. Then we use these
representations as a baseline to learn the relevant representations for $D_2$. 
Their work provide guidelines to unsupervised representation learning and 
experiments on classification dataset. We extend their work to semantic segmentation
for tasks where similar content representation naturally arises across distinct
input distributions. This solves the challenge of the lack of annotated 
examples.

\section{Method}

\subsection{Training}
Figure \ref{fig:method} illustrates the training pipeline: the upper network
has been trained in a supervised manner on an annotated dataset $D_1$. 
It has learned relevant high-level semantic representations of the image 
content and is then frozen (grey). The bottom network is initialised with the
weights of the upper one and is trained on  
the second dataset $D_2$. $D_2$ has content similar to $D_1$ but with a different 
distribution. Given two images $(X_1, X_2) \in D_1 \times D_2$,  
the blue layers are fine-tuned to generate the same representation 
on $X_2$ as the upper network on $X_1$. 
The training is a regression between feature maps and does not require 
any annotation on $D_2$. 

More formally, let $H(\theta)$ be a network model parameterised by weights
$\theta$ and ${\cal F}(\theta) = \left\{F^l(\theta), \;l \in L\right\}$ a set 
of feature maps of the network $H(\theta)$. 
For example in VGG-16, ${\cal F}(\theta)$ could be the {\tt conv1\_1} or {\tt conv2\_2} layer. 
Let $H_1 = H(\theta_1)$ and $H_2 = H(\theta_2)$ be two instances of $H$ trained on $D_1$ 
and $D_2$, leading respectively to the weights $\theta_1$ and $\theta_2$. 
$H_1$ is trained with the pixel-wise cross-entropy loss. 
$H_2$ first initialised with the weights $\theta_1$ and then trained on the set of image pairs 
$\left\{(X_1^i, X_2^i), \; i \in I\right\} \subset D_1 \times D_2$. Each pair has a similar content but 
different pixel distributions. However, $H_2$ is not trained using the cross-entropy
loss, but it is trained so that its feature maps 
${\cal F}(\theta_2) = \left\{F^l_2,\;l \in L\right\}$
match the corresponding feature maps ${\cal F}(\theta_1)=\left\{F^l_1,\;l \in L\right\}$ from $H_1$. 
Given a set of feature map ${\cal F}$, for one image pair, the training loss is:

\begin{equation}
  \begin{split}
    \mathcal{L}(H_2)  & = \sum_{l \in L}{w_l \mathcal{L}(F^l_2)}\\
                              & = \sum_{l \in L}{ w_l \| F^l_1 - F^l_2 \|^2}
  \end{split}
\end{equation}

Each feature map $F^l_2$ generates a loss weighted by $w_l$. 
During training, the loss from the $l$-th feature map is back-propagated only 
in the layers lower than $l$. In our implementation, the feature weights has
been kept constant. Investigations over more complex weighing strategies have been left
for future work.

\subsection{Evaluation}

Following the standard practice in segmentation problems, we evaluate our method with the segmentation performance of $H_2$ on 
the test set of $D_2$. We use both the mean class accuracy (acc) and the mean 
Intersection Over Union (IOU).
The performances of $H_2$ are compared to three baselines. 
The baseline B0 
measures the performance of $H_1$ on $D_2$, i.e. how well $H_1$ 
generalises to a dataset with the same content but a different pixel distribution. 
This also gives a quantitative measure of the transformation between two datasets.
The transformation is small when the performances of $H_1$ on $D_1$ and $D_2$ are equivalent.
The baseline B1 measures the performance of $H_2$ trained from scratch 
on $D_2$ when 
pixel-wise annotations are provided. This ideal setting provides the performance 
the transfer learning should aim at.
The last baseline B2 measures the performance of $H_2$ when it is initialised 
with $\theta_1$ and then fine-tuned on $D_2$ using standard cross-entropy loss with full annotations. This provides the performance of 
classic supervised fine-tuning our method should reach, even though it does not require explicit annotations. The baselines are summarised in Table \ref{tab:baselines}.

\begin{table}
\label{tab:baselines}
\begin{center}
\begin{tabular}{|l|l|l|}
\hline
  B & Training & Test \\
\hline
  B0 & $H_1$ on annotated $D_1$ & $H_1$ on $D_2$ \\
\hline
  B1 & $H_2$ on annotated $D_2$ & $H_2$ on $D_2$ \\
\hline
  B2 & $H_2$ initialised with $\theta_1$,  & $H_2$ on $D_2$ \\
     & fine-tuned on annotated $D_2$ &                         \\
\hline          
\end{tabular}
\end{center}
  \caption{Baselines summary.} 
\end{table}

\subsection{Visualisation}
Once $H_2$ is trained, we propose to visualise the changes of representation space induced
by the regression. In principle, this is achieved by feeding an image $X_1 \in D_1$ to $H_2$ 
which generates a set of features ${\cal F}_2(X_1)$.  We then need to design an optimisation for an image $X$ initialised with noise so that ${\cal F}_2(X)$ 
matches ${\cal F}_2(X_1)$. 
At the end of the optimisation, $X$ should display the same content as $X_1$ with the 
style of $D_2$. 

To generate these images, we adapt the feature map inversion method from
\cite{Mahendran_2015_CVPR} to integrate style transfer into the inversion. In
addition, we adapt the style transfer method from \cite{gatys2016image},
\cite{johnson2016perceptual} to generate image content and style from only one
network instead of two. 

In practice, the content and style generation is performed as follows: we feed $X$ to $H_2$, 
compute a content loss and a style loss between ${\cal F}_2(X_1)$ 
and ${\cal F}_2(X)$. 
Then we use Stochastic Gradient Descent (SGD)
on $X$ so as to make the features extracted from $X$ converge to those extracted
from $X_1$.

The loss used in the optimisation is computed for each feature map $F^l_2$. We use the loss definitions and Gram matrix definition 
from \cite{gatys2016image}. The content loss is a simple $\mathcal{L}_2$ loss
between feature maps. The style loss is a normalised $\mathcal{L}_2$ loss 
between the Gram matrix of the feature maps. This matrix is computed from a 2D
representation of each $F^l_2$.

\begin{equation}
  \begin{split}
    \mathcal{L}_{content}(l)  &= \frac{1}{2} \|F^l_2(X_1) - F^l_2(X)\|^2 \\ 
    \mathcal{L}_{style}(l)    &\propto \|G(F^l_2(X_1))-G(F^l_2(X))\|^2 \\ 
  \end{split}
\end{equation}

\section{Experiments}

\subsection{Dataset}

\begin{figure}[thb]
  \centering
  \includegraphics[width=\linewidth]{transfo_voc12_2_244.png}
  \caption{Synthetic transformations. Left-right: VOC, photocopy, ripple, cubism}
  \label{fig:transfoVOC12}
\end{figure}

To compute the quantitative baselines, we use the PASCAL VOC12 
\cite{everingham2015pascal} with three synthetic transformations.  
It contains
20 categories to segment and classify from an additional background category. 
The original dataset contains 1464 train images, 
1449 validation images and 1456 test images. \cite{hariharan2011semantic} provides additional
annotated images resulting in 10 582 training images. 
Three transformations are simulated on this dataset using
GIMP\footnote{\url{https://www.gimp.org/}} and illustrated
in Figure \ref{fig:transfoVOC12}. The regression is trained on the 10 582 original
images together with their transformation. $H_2$ is then evaluated on the transformed validation 
set resulting from each transformation.

$T_1$ is generated using the 'photocopy' filter 
which simulates a change of domain color with a change of saturation. This can
be encountered in environment monitoring across long period of time such as
land usage monitoring \cite{richard2018}.
$T_2$ is generated with the ripple distortion to simulate misalignment of image 
content together with noise. This is typical in natural environment monitoring such 
as in the dataset from \cite{griffith2017symphony}. 
$T_3$ is generated with the cubism filter to simulate both a change of texture, 
misalignment with edge noise. 

We measure the distance between the original 
PASCAL VOC 12 dataset $D_1$ and a transformed dataset $D_2$ (Table \ref{tab:dd}) 
with the normalized performance degradation of the network 
$H_1$ between $D_1$ and $D_2$. We train $H_1$ on $D_1$ following the setup from 
DeepLab V3 \cite{chen2018deeplab} and reach an accuracy of 79.92 and a mIOU of 
69.22. Let $acc_2$ and $mIOU_2$ the performance of $H_1$ on $D_2$, the distance 
is computed as follow: 
\begin{equation}
  d(D_1, D_2) = \frac{1}{2}( \frac{\|79.92 - acc_2\|}{79.92} + 
  \frac{\|69.22 - mIOU_2\|}{69.22})
\end{equation}

\begin{table}
\begin{center}
\begin{tabular}{|c|c|c|}
\hline
T1: Photocopy & T2: Ripple & T3: Cubism\\
\hline     
32.48  & 62.59  & 94.03\\
\hline
\end{tabular}
\end{center}
\caption{Dataset distance and measure of the transformation
    complexity.}
\label{tab:dd}
\end{table}

From the normalized distances shown in Table~\ref{tab:dd}, it is clear that
the three transformation exhibit a increasing level of complexity that
challenges a network trained only on $D_1$. 

\begin{figure}[thb]
  \centering
  \includegraphics[width=\linewidth]{feat_map_conv.png}
   \caption{Feature map convergence. 
   Line 1 show the original images. 
   Line 2 show the feature map of $H_1$ on $D_1$. 
   Line 3 shows the transformed image. 
   Line 4 show the feature map of $H_1$ on $D_2$. 
   Line 5 show the feature map of $H_2$ on $D_2$.}
  \label{fig:feat_map}
\end{figure}
\subsection{Experimental Setup}

\paragraph{Supervised training of $H_1$.}
$H$ follows the VGG-16 architecture \cite{simonyan2014very} from DeepLabV3 \cite{chen2018deeplab} for
training time considerations. The network reaches near state-of-the-art
performance on PASCAL VOC 12 with only 5 hours of training on an NVIDIA Ge1080Ti.
The training setting is the same as in \cite{chen2018deeplab}:
we train the network for 20 000 iterations with a batch size of 10, stochastic
gradient descent with a momentum of 0.9, a weight decay of 0.5 and the "poly"
learning rate policy initialized at $2.5 \times 10^{-4}$ with
$power=0.9$. 

\paragraph{Feature map training of $H_2$.}
$H_2$ has the same architecture as $H_1$. We feed $D_1$ to the trained $H_1$,
$D_2$ to $H_2(\theta)$ and train a regression between a set of feature map ${\cal F}(\theta)$. 
We chose several sets of feature maps to build ${\cal F}$ to better understand the hierarchical representation
model of the network. An intuition gathered from the literature 
(\cite{dosovitskiy2016inverting,oquab2014learning,simonyan2013deep}) suggests that early layers captures low-level 
representations such as colors and edges, and deeper-level more complex representations 
such as object contours and their label. 
We evaluate this intuition by comparing the performance of $H_2$ with regression 
on early layers such as {\tt pool\_1} or {\tt pool\_2} versus deeper layers 
such as {\tt pool\_5}. 
We also evaluate the correlation between these feature maps with regression 
on several layers simultaneously: {\tt pool\_1} to {\tt pool\_5}. 
Finally, we evaluate the relative importance of the features maps with two weights 
strategies. In the first one, $w_l$ increases with the layer level, i.e. $w_l$ is 
higher for {\tt pool\_5} than for {\tt pool\_1}. In the second, the weights are reversed. We use a uniform weight distribution [0.2, 0.4, 0.6, 0.8, 0.9]. 


\subsection{Experimental Results}

\begin{figure}[thb]
  \centering
  \includegraphics[width=\linewidth]{rec_pool1_133.png}
  \caption{Style reconstruction: $H_2$ is fed with the left-most image and generates a feature map. Then an image is generated to match this feature map. Left-right: $H_2$ trained on $T_1, T_2, T_3$. The generated image match the content and the style transformation after 2000 iterations.}
  \label{fig:rec_pool5_53}
\end{figure}

Our method reaches comparable or better performance than fine-tuning on all transformation 
(Table~\ref{tab:res_pool5}). Let us recall however, that this is achieved without using an explicitly labeled segmentation on $D_2$.
Figure \ref{fig:feat_map} shows that the deep representations of $H_2$ on $D_2$ 
(bottom lines) converge towards the deep representation of $H_1$ on $D_1$ (Figure \ref{fig:feat_map} line 2).
Visualization (Figure \ref{fig:rec_pool5_53}) shows that this transfer learning method 
adapt the parameters to the dataset style. Reconstruction from the feature map 
shows the original image with the appropriate style.  

\begin{table}[tbh]
\begin{center}
\begin{tabular}{|l|c|c|c|}
\hline
B  & T1 & T2 & T3 \\
\hline
B0  &54.73-46.07&31.76-24.28&5.18-3.78\\
\hline
B1  &74.43-62.26&73.43-62.58&54.04-43.74\\
\hline
B2  &\textbf{74.65}-63.41&74.71-63.99&\textbf{55.39}-45.37\\
\hline
Our &74.05-\textbf{64.39}&\textbf{78.65-67.23} &54.46-\textbf{45.79}\\
\hline
\end{tabular}
\end{center}
  \caption{Segmentation perf. of $H_2$ on $D_2$ for regression on pool\_5 (acc-mIOU)}
\label{tab:res_pool5}
\end{table}

Table \ref{tab:res_pool5} shows that our method reaches performances similar to 
classic fine-tuning for all image transformations. 
The first line (B0) shows the performance of $H_1$ on the transformed dataset. 
$H_1$ is trained with supervision on the original dataset $D_1$. So these metrics show the generalization
properties of $H_1$ on the transformed dataset. It also reflects the level of transformation between the original dataset
and the transformed ones. For example, the photocopy dataset distribution is closer to the 
original dataset than the cubism dataset distribution: the accuracy drops from 54.73 to 5.18. 
Our method presents an improvement over all B0 results which means that 
transferring deep representations is a relevant transfer learning method for semantic segmentation.
The second and third lines are the baselines we compare our method to: baseline B1 shows the performance 
of the network if trained from scratch  on $D_2$ when pixel-wise annotations
are provided. 
Baseline B2 shows the performance of $H_2$ when it is finetuned from $H(\theta_1)$ with annotations of $D_2$. 
B2 outperforms B1 which confirms that transfer learning is relevant to DCNN
even when pixel-wise annotations are provided. 
The third line shows our method always outperforms B2 with respect to mIOU 
and either outperform or reach comparable mean accuracy as B2. This shows that 
transfer learning of deep representations can replace classic DCNN finetuning.

\begin{table}
\begin{center}
\begin{tabular}{|l|c|c|c|}
\hline
T       & T1              & T2     & T3 \\
\hline
  \texttt{pool\_1}   &60.77-50.40 &24.13-16.47 &5.32-3.77\\
\hline
  \texttt{pool\_2}   &65.84-54.26 &25.59-17.72 &7.03-5.55\\
\hline
  \texttt{pool\_3} &57.14-46.47   &25.56-17.70   &4.76-3.45 \\
\hline
  \texttt{pool\_4}  &4.76-3.45    &25.53-17.67  &43.25-35-14\\
\hline
  \texttt{pool\_5}   &\textbf{75.05-64.39}&\textbf{78.65-67.23}&\textbf{54.47-45.79} \\
\hline
  $W_{inc}$ &69.27-59.15 & 4.76-3.45 & 35.24-29.44 \\
\hline 
  $W_{dec}$ &71.80-61.71 & 4.76-0.03 & 4.76-3.45 \\
\hline
\end{tabular}
\end{center}
\caption{Segmentation perf. of $H_2$ on $D_2$ for regression on different feature maps}
\label{tab:feat_res}
\end{table}

Table \ref{tab:feat_res} shows the segmentation performance of $H_2$ when trained 
with different regression setting. The best performances are reached
when the transfer learning
is supervised between only the layers \texttt{pool\_5} of each network. 
This suggests that
 the high level representation are the most relevant learning to transfer for 
semantic segmentation. 
The feature maps relevant to transfer may change with the nature of 
the transformation. The experiments do not provide a generic rule on
the choice of the feature maps to transfer. But we can still draw an intuition with 
the results in Table \ref{tab:feat_res}. 
For $T_1$, we observe that transferring layers up to {\tt pool\_3} trains $H_2$ to 
improve compared to B0 which is not the case for $T_2$ and $T_3$. One explanation 
can be that $T_1$ is only a texture transformation which conserves the alignment 
of the image. So the only change in the dataset is with regard to low-level
features usually generated in early layers. However, in $T_2$ and $T_3$, the 
contours of the semantic units in the images change. In $T_2$ it is a regular 
change defined by the ripple parameters and for $T_3$ the contour pattern is 
random. Object contours are features usually generated in deeper layers which 
can explain why the transfer of \texttt{pool\_5} only is relevant for these datasets.
The results also suggest that when low-level layers are not relevant to transfer, they
can hinder the transfer of the relevant layers. For example, the 
transfer learning on multiple layers performs worse than the transfer on \texttt{pool\_5} only. 
For $T_3$, we observe that a weight distribution that favors high layers performs
better than $W_{dec}$ but worse than the transfer of \texttt{pool\_5}. 
Future work will focus on the influence of the choice of the features maps in 
the transfer learning performance together with the choice of weight distribution. 

\section{Conclusion}
This work extends the transfer learning of deep representations initiated 
in \cite{bengio2012deep} to semantic segmentation. The application to recurrent
dataset overcomes one of its limitation which is the lack of instance pairs
with similar deep representations. For these applications, such pairs arise 
naturally.
This method shows similar performance to classic fine-tuning without using 
annotations on the target task. The preliminary results suggest that 
the transfer learning is the most relevant between high layers. 
While this proof-of-concept targets the feasibility of this method, the
experiments are not extensive enough to draw a relation between the transfer
learning performance and the level of transferred deep representations. 
In future work, we will explore this dependence together with the correlation 
of transfer learning on multiple deep representations simultaneously.

\bibliography{bmvc_review}
\end{document}




