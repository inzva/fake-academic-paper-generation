
%% bare_conf_compsoc.tex
%% V1.4b
%% 2015/08/26
%% by Michael Shell
%% See:
%% http://www.michaelshell.org/
%% for current contact information.
%%
%% This is a skeleton file demonstrating the use of IEEEtran.cls
%% (requires IEEEtran.cls version 1.8b or later) with an IEEE Computer
%% Society conference paper.
%%
%% Support sites:
%% http://www.michaelshell.org/tex/ieeetran/
%% http://www.ctan.org/pkg/ieeetran
%% and
%% http://www.ieee.org/

%%*************************************************************************
%% Legal Notice:
%% This code is offered as-is without any warranty either expressed or
%% implied; without even the implied warranty of MERCHANTABILITY or
%% FITNESS FOR A PARTICULAR PURPOSE!
%% User assumes all risk.
%% In no event shall the IEEE or any contributor to this code be liable for
%% any damages or losses, including, but not limited to, incidental,
%% consequential, or any other damages, resulting from the use or misuse
%% of any information contained here.
%%
%% All comments are the opinions of their respective authors and are not
%% necessarily endorsed by the IEEE.
%%
%% This work is distributed under the LaTeX Project Public License (LPPL)
%% ( http://www.latex-project.org/ ) version 1.3, and may be freely used,
%% distributed and modified. A copy of the LPPL, version 1.3, is included
%% in the base LaTeX documentation of all distributions of LaTeX released
%% 2003/12/01 or later.
%% Retain all contribution notices and credits.
%% ** Modified files should be clearly indicated as such, including  **
%% ** renaming them and changing author support contact information. **
%%*************************************************************************


% *** Authors should verify (and, if needed, correct) their LaTeX system  ***
% *** with the testflow diagnostic prior to trusting their LaTeX platform ***
% *** with production work. The IEEE's font choices and paper sizes can   ***
% *** trigger bugs that do not appear when using other class files.       ***                          ***
% The testflow support page is at:
% http://www.michaelshell.org/tex/testflow/



\documentclass[conference,compsoc]{IEEEtran}
% Some/most Computer Society conferences require the compsoc mode option,
% but others may want the standard conference format.
%
% If IEEEtran.cls has not been installed into the LaTeX system files,
% manually specify the path to it like:
% \documentclass[conference,compsoc]{../sty/IEEEtran}





% Some very useful LaTeX packages include:
% (uncomment the ones you want to load)


% *** MISC UTILITY PACKAGES ***
%
%\usepackage{ifpdf}
% Heiko Oberdiek's ifpdf.sty is very useful if you need conditional
% compilation based on whether the output is pdf or dvi.
% usage:
% \ifpdf
%   % pdf code
% \else
%   % dvi code
% \fi
% The latest version of ifpdf.sty can be obtained from:
% http://www.ctan.org/pkg/ifpdf
% Also, note that IEEEtran.cls V1.7 and later provides a builtin
% \ifCLASSINFOpdf conditional that works the same way.
% When switching from latex to pdflatex and vice-versa, the compiler may
% have to be run twice to clear warning/error messages.


% *** CITATION PACKAGES ***
%
%\ifCLASSOPTIONcompsoc
%  % IEEE Computer Society needs nocompress option
%  % requires cite.sty v4.0 or later (November 2003)
%  \usepackage[nocompress]{cite}
%\else
%  % normal IEEE
%  \usepackage{cite}
%\fi
% cite.sty was written by Donald Arseneau
% V1.6 and later of IEEEtran pre-defines the format of the cite.sty package
% \cite{} output to follow that of the IEEE. Loading the cite package will
% result in citation numbers being automatically sorted and properly
% "compressed/ranged". e.g., [1], [9], [2], [7], [5], [6] without using
% cite.sty will become [1], [2], [5]--[7], [9] using cite.sty. cite.sty's
% \cite will automatically add leading space, if needed. Use cite.sty's
% noadjust option (cite.sty V3.8 and later) if you want to turn this off
% such as if a citation ever needs to be enclosed in parenthesis.
% cite.sty is already installed on most LaTeX systems. Be sure and use
% version 5.0 (2009-03-20) and later if using hyperref.sty.
% The latest version can be obtained at:
% http://www.ctan.org/pkg/cite
% The documentation is contained in the cite.sty file itself.
%
% Note that some packages require special options to format as the Computer
% Society requires. In particular, Computer Society  papers do not use
% compressed citation ranges as is done in typical IEEE papers
% (e.g., [1]-[4]). Instead, they list every citation separately in order
% (e.g., [1], [2], [3], [4]). To get the latter we need to load the cite
% package with the nocompress option which is supported by cite.sty v4.0
% and later.





% *** GRAPHICS RELATED PACKAGES ***
%
\ifCLASSINFOpdf
  % \usepackage[pdftex]{graphicx}
  % declare the path(s) where your graphic files are
  % \graphicspath{{../pdf/}{../jpeg/}}
  % and their extensions so you won't have to specify these with
  % every instance of \includegraphics
  % \DeclareGraphicsExtensions{.pdf,.jpeg,.png}
\else
  % or other class option (dvipsone, dvipdf, if not using dvips). graphicx
  % will default to the driver specified in the system graphics.cfg if no
  % driver is specified.
  % \usepackage[dvips]{graphicx}
  % declare the path(s) where your graphic files are
  % \graphicspath{{../eps/}}
  % and their extensions so you won't have to specify these with
  % every instance of \includegraphics
  % \DeclareGraphicsExtensions{.eps}
\fi
% graphicx was written by David Carlisle and Sebastian Rahtz. It is
% required if you want graphics, photos, etc. graphicx.sty is already
% installed on most LaTeX systems. The latest version and documentation
% can be obtained at:
% http://www.ctan.org/pkg/graphicx
% Another good source of documentation is "Using Imported Graphics in
% LaTeX2e" by Keith Reckdahl which can be found at:
% http://www.ctan.org/pkg/epslatex
%
% latex, and pdflatex in dvi mode, support graphics in encapsulated
% postscript (.eps) format. pdflatex in pdf mode supports graphics
% in .pdf, .jpeg, .png and .mps (metapost) formats. Users should ensure
% that all non-photo figures use a vector format (.eps, .pdf, .mps) and
% not a bitmapped formats (.jpeg, .png). The IEEE frowns on bitmapped formats
% which can result in "jaggedy"/blurry rendering of lines and letters as
% well as large increases in file sizes.
%
% You can find documentation about the pdfTeX application at:
% http://www.tug.org/applications/pdftex





% *** MATH PACKAGES ***
%
%\usepackage{amsmath}
% A popular package from the American Mathematical Society that provides
% many useful and powerful commands for dealing with mathematics.
%
% Note that the amsmath package sets \interdisplaylinepenalty to 10000
% thus preventing page breaks from occurring within multiline equations. Use:
%\interdisplaylinepenalty=2500
% after loading amsmath to restore such page breaks as IEEEtran.cls normally
% does. amsmath.sty is already installed on most LaTeX systems. The latest
% version and documentation can be obtained at:
% http://www.ctan.org/pkg/amsmath





% *** SPECIALIZED LIST PACKAGES ***
%
%\usepackage{algorithmic}
% algorithmic.sty was written by Peter Williams and Rogerio Brito.
% This package provides an algorithmic environment fo describing algorithms.
% You can use the algorithmic environment in-text or within a figure
% environment to provide for a floating algorithm. Do NOT use the algorithm
% floating environment provided by algorithm.sty (by the same authors) or
% algorithm2e.sty (by Christophe Fiorio) as the IEEE does not use dedicated
% algorithm float types and packages that provide these will not provide
% correct IEEE style captions. The latest version and documentation of
% algorithmic.sty can be obtained at:
% http://www.ctan.org/pkg/algorithms
% Also of interest may be the (relatively newer and more customizable)
% algorithmicx.sty package by Szasz Janos:
% http://www.ctan.org/pkg/algorithmicx




% *** ALIGNMENT PACKAGES ***
%
%\usepackage{array}
% Frank Mittelbach's and David Carlisle's array.sty patches and improves
% the standard LaTeX2e array and tabular environments to provide better
% appearance and additional user controls. As the default LaTeX2e table
% generation code is lacking to the point of almost being broken with
% respect to the quality of the end results, all users are strongly
% advised to use an enhanced (at the very least that provided by array.sty)
% set of table tools. array.sty is already installed on most systems. The
% latest version and documentation can be obtained at:
% http://www.ctan.org/pkg/array


% IEEEtran contains the IEEEeqnarray family of commands that can be used to
% generate multiline equations as well as matrices, tables, etc., of high
% quality.



% *** SUBFIGURE PACKAGES ***
%\ifCLASSOPTIONcompsoc
%  \usepackage[caption=false,font=footnotesize,labelfont=sf,textfont=sf]{subfig}
%\else
%  \usepackage[caption=false,font=footnotesize]{subfig}
%\fi
% subfig.sty, written by Steven Douglas Cochran, is the modern replacement
% for subfigure.sty, the latter of which is no longer maintained and is
% incompatible with some LaTeX packages including fixltx2e. However,
% subfig.sty requires and automatically loads Axel Sommerfeldt's caption.sty
% which will override IEEEtran.cls' handling of captions and this will result
% in non-IEEE style figure/table captions. To prevent this problem, be sure
% and invoke subfig.sty's "caption=false" package option (available since
% subfig.sty version 1.3, 2005/06/28) as this is will preserve IEEEtran.cls
% handling of captions.
% Note that the Computer Society format requires a sans serif font rather
% than the serif font used in traditional IEEE formatting and thus the need
% to invoke different subfig.sty package options depending on whether
% compsoc mode has been enabled.
%
% The latest version and documentation of subfig.sty can be obtained at:
% http://www.ctan.org/pkg/subfig




% *** FLOAT PACKAGES ***
%
%\usepackage{fixltx2e}
% fixltx2e, the successor to the earlier fix2col.sty, was written by
% Frank Mittelbach and David Carlisle. This package corrects a few problems
% in the LaTeX2e kernel, the most notable of which is that in current
% LaTeX2e releases, the ordering of single and double column floats is not
% guaranteed to be preserved. Thus, an unpatched LaTeX2e can allow a
% single column figure to be placed prior to an earlier double column
% figure.
% Be aware that LaTeX2e kernels dated 2015 and later have fixltx2e.sty's
% corrections already built into the system in which case a warning will
% be issued if an attempt is made to load fixltx2e.sty as it is no longer
% needed.
% The latest version and documentation can be found at:
% http://www.ctan.org/pkg/fixltx2e


%\usepackage{stfloats}
% stfloats.sty was written by Sigitas Tolusis. This package gives LaTeX2e
% the ability to do double column floats at the bottom of the page as well
% as the top. (e.g., "\begin{figure*}[!b]" is not normally possible in
% LaTeX2e). It also provides a command:
%\fnbelowfloat
% to enable the placement of footnotes below bottom floats (the standard
% LaTeX2e kernel puts them above bottom floats). This is an invasive package
% which rewrites many portions of the LaTeX2e float routines. It may not work
% with other packages that modify the LaTeX2e float routines. The latest
% version and documentation can be obtained at:
% http://www.ctan.org/pkg/stfloats
% Do not use the stfloats baselinefloat ability as the IEEE does not allow
% \baselineskip to stretch. Authors submitting work to the IEEE should note
% that the IEEE rarely uses double column equations and that authors should try
% to avoid such use. Do not be tempted to use the cuted.sty or midfloat.sty
% packages (also by Sigitas Tolusis) as the IEEE does not format its papers in
% such ways.
% Do not attempt to use stfloats with fixltx2e as they are incompatible.
% Instead, use Morten Hogholm'a dblfloatfix which combines the features
% of both fixltx2e and stfloats:
%
% \usepackage{dblfloatfix}
% The latest version can be found at:
% http://www.ctan.org/pkg/dblfloatfix




% *** PDF, URL AND HYPERLINK PACKAGES ***
%
%\usepackage{url}
% url.sty was written by Donald Arseneau. It provides better support for
% handling and breaking URLs. url.sty is already installed on most LaTeX
% systems. The latest version and documentation can be obtained at:
% http://www.ctan.org/pkg/url
% Basically, \url{my_url_here}.




% *** Do not adjust lengths that control margins, column widths, etc. ***
% *** Do not use packages that alter fonts (such as pslatex).         ***
% There should be no need to do such things with IEEEtran.cls V1.6 and later.
% (Unless specifically asked to do so by the journal or conference you plan
% to submit to, of course. )
\usepackage{ulem}
\usepackage{amsmath}
\usepackage{mathrsfs}
\usepackage{hyperref}       % hyperlinks
\usepackage{url}            % simple URL typesetting


\usepackage{graphicx}
\usepackage{graphics}
\graphicspath{{images/}}
%\usepackage{ulem}
\usepackage{booktabs}
\itshape

\small
%\bibliographystyle{plain}
\bibliographystyle{unsrt}
%\bibliographystyle{ieeetr}
%\bibliographystyle{IEEEtran}
\newcommand\embf{\textbf}

%参考文献排序、压缩
\usepackage[numbers,sort&compress]{natbib}

%可解决论文最后一页左右两列不平衡的问题
\usepackage{flushend}


% correct bad hyphenation here
\hyphenation{op-tical net-works semi-conduc-tor}


\begin{document}

% begin content
%\tableofcontents
%\pagenumbering{arabic}
% end content

%
% paper title
% Titles are generally capitalized except for words such as a, an, and, as,
% at, but, by, for, in, nor, of, on, or, the, to and up, which are usually
% not capitalized unless they are the first or last word of the title.
% Linebreaks \\ can be used within to get better formatting as desired.
% Do not put math or special symbols in the title.
\title{High-Quality Face Image Super-Resolution Using \\ Conditional Generative Adversarial Networks}


% author names and affiliations
% use a multiple column layout for up to three different
% affiliations
\author{\IEEEauthorblockN{Huang Bin and Chen Weihai and Wu Xingming}
\IEEEauthorblockA{School of Automation Science \\and Electrical Engineering,
Beihang University\\
XueYuan Road No.37, HaiDian District, Beijing, China\\
Email: marshuangbin@buaa.edu.cn\\
Email:whchenbuaa@126.com\\
Email:wxmbuaa@163.com}
%\and
%\IEEEauthorblockN{Chen Weihai}
%\IEEEauthorblockA{School of Automation Science \\and Electrical Engineering\\
%Beihang University\\
%XueYuan Road No.37, \\HaiDian District, Beijing, China\\
%Email:whchenbuaa@126.com}
%\and
%\IEEEauthorblockN{Wu Xingming}
%\IEEEauthorblockA{School of Automation Science \\and Electrical Engineering\\
%Beihang University\\
%XueYuan Road No.37, \\HaiDian District, Beijing, China\\
%Email:wxmbuaa@163.com}
\and
\IEEEauthorblockN{Lin Chun-Liang}
\IEEEauthorblockA{Department of Electrical Engineering,
\\ National Chung Hsing University\\
145 Xingda Rd., South Dist.\\Taichung City 402, Taiwan \\
Email:chunlin@dragon.nchu.edu.tw}

}

% conference papers do not typically use \thanks and this command
% is locked out in conference mode. If really needed, such as for
% the acknowledgment of grants, issue a \IEEEoverridecommandlockouts
% after \documentclass

% for over three affiliations, or if they all won't fit within the width
% of the page (and note that there is less available width in this regard for
% compsoc conferences compared to traditional conferences), use this
% alternative format:
%
%\author{\IEEEauthorblockN{Michael Shell\IEEEauthorrefmark{1},
%Homer Simpson\IEEEauthorrefmark{2},
%James Kirk\IEEEauthorrefmark{3},
%Montgomery Scott\IEEEauthorrefmark{3} and
%Eldon Tyrell\IEEEauthorrefmark{4}}
%\IEEEauthorblockA{\IEEEauthorrefmark{1}School of Electrical and Computer Engineering\\
%Georgia Institute of Technology,
%Atlanta, Georgia 30332--0250\\ Email: see http://www.michaelshell.org/contact.html}
%\IEEEauthorblockA{\IEEEauthorrefmark{2}Twentieth Century Fox, Springfield, USA\\
%Email: homer@thesimpsons.com}
%\IEEEauthorblockA{\IEEEauthorrefmark{3}Starfleet Academy, San Francisco, California 96678-2391\\
%Telephone: (800) 555--1212, Fax: (888) 555--1212}
%\IEEEauthorblockA{\IEEEauthorrefmark{4}Tyrell Inc., 123 Replicant Street, Los Angeles, California 90210--4321}}




% use for special paper notices
%\IEEEspecialpapernotice{(Invited Paper)}




% make the title area
\maketitle

% As a general rule, do not put math, special symbols or citations
% in the abstract
\begin{abstract}
We propose a novel single face image super-resolution method, which named \textit{Face Conditional Generative Adversarial Network}(FCGAN), based on boundary equilibrium generative adversarial networks. Without taking any facial prior information, our method can generate a high-resolution face image from a low-resolution one. Compared with existing studies, both our training and testing phases are end-to-end pipeline with little pre/post-processing. To enhance the convergence speed and strengthen feature propagation, skip-layer connection is further employed in the generative and discriminative networks. Extensive experiments demonstrate that our model achieves competitive performance compared with state-of-the-art models.
\end{abstract}

% no keywords


% For peer review papers, you can put extra information on the cover
% page as needed:
% \ifCLASSOPTIONpeerreview
% \begin{center} \bfseries EDICS Category: 3-BBND \end{center}
% \fi
%
% For peerreview papers, this IEEEtran command inserts a page break and
% creates the second title. It will be ignored for other modes.
\IEEEpeerreviewmaketitle


\section{Introduction}
Single image super resolution(SISR), a greatly challenging task of computer vision and machine learning, attempts to reconstruct a high-resolution(HR) image from a low-resolution(LR) image. Super resolution(SR) is commonly divided into two categories based on their tasks, namely generic image SR and class-specific image SR. The former takes little class information into account, which aims to recover any kinds of high resolution image from corresponding low-resolution image. In general, the latter usually refers to face image super resolution or face hallucination if the class is face.

Face image super resolution or face hallucination\citep{Jia2011Fast, zhu2016deep, wang2014comprehensive, li-PR2014face, autee2015review, Jiang2016Noise-TCYB, jin2015robust, su2016supervised, W2016DeepJFHR} is an important branch of super-resolution(SR). The great distinction between the both techniques is that face hallucination always employs typical facial priors (eg. face spatial configuration and facial landmark detection) with strong cohesion to face domain concept. More realistic and sharper details, which plays a crucial role in intelligence surveillance\cite{Jia2011Fast, wang2014comprehensive} and face recognition\cite{W2016DeepJFHR}, are taken by HR face images than corresponding LR images. Due to long distance imaging, the limitations on storage and low-cost electronic imaging systems, LR images appear in many cases instead of HR images. Thus, SR has turned out to be an active research filed in the past few years.

Face image SR is an ill-posed problem (as same as generic image SR), for which it needs to recover 16 pixels (for $4\times$ upscaling factors) from each given pixel. While, recent years have witnessed a tremendous growth of research and development in the field, in particular using learning-based methods.
\begin{figure}[h]
  \centering
  \includegraphics[width = 8.5cm]{GAN_structure3.jpg}
  \caption{The pipeline of FCGAN. The architecture of generator and discriminator network with corresponding filter size and output channels(C) for each convolutional layer. In the testing phase, only the generator network is employed and the discriminator network does not work. }
  \label{Fig:architecture}
\end{figure}

In this paper, we propose a HR face image framework ($4\times$ upscaling factors) based on boundary equilibrium generative adversarial network(BEGAN)\cite{David-BEGAN2017}. In order to adapt BEGAN for SR task, single low-resolution face image is considered as the prior condition to generate a high-resolution one. So, we refer to the framework as \textit{Face Conditional Generative Adversarial Network}(named FCGAN for short hereafter). Our proposed method does not utilize any priors on face structure or face spatial configuration. In addition, it is also an end-to-end solution to generate HR face images without need any pre-trained model. We perform extensive experiments, which demonstrates that our method not only achieves high Peak Signal to Noise Ratio(PSNR), but also improves actual visual quality.

Overall, the contributions of this paper are mainly in three aspects:

\begin{itemize}
  \item We propose a novel end-to-end method (FCGAN), with $4\times$ upscaling factors, to learn mapping between low-resolution single face images to high-resolution one. The method can robustly generate a high-quality face image from low-resolution one.
  \item To the best of our knowledge, our method is the first attempt to develop BEGAN\cite{David-BEGAN2017} to generate HR face images from low-resolution ones regardless of pose, facial expressions variation, face alignment and lighting. Our model considers a low-resolution image $I^{LR}$ as the input instead of random noise.
  \item We introduce the pixel-wise $L_1$ loss function to optimize the generative and discriminative models. Compared with state-of-the-art models, extensive experiments show that FCGAN achieve competitive performance on both visual quality and quantitative analysis.

\end{itemize}

%%===========================end introduction==================================%%

\section{Related work}

In general, image SR methods can be classified into three categories: interpolation methods, reconstruction-based methods, and example (learning)-based methods. Among them, due to the simply pipeline and excellent performance, the example-based methods\cite{zhu2016deep, Jiang2016Noise-TCYB, jin2015robust, W2016DeepJFHR, Christian-SRGAN-CVPR2017, dong2016accelerating, Dong-He-PAMI2016Image, Jiang2016SRLSP, Kim_2016_DRCN, LapSRN_CVPR2017, Yu-URGAN-ECCV2016} achieve explosive development in the past years. In this section, we will also mostly focus on discussion example-based methods.

\subsection{generic image SR}
In the past few years, Deep convolutional neural networks(DCNNs) have demonstrated outstanding performance in single image SR. Dong et al.'s work\cite{Dong-He-PAMI2016Image} first extend CNN to the field of image SR and demonstrate that deep learning can achieve higher quality image than other learning-based methods. The authors design a simple fully convolutional neural network that directly learns an end-to-end mapping between low-resolution and high-resolution images. Furthermore, they point out that the three convolutional layers can be abstracted into patch extraction and representation, non-linear mapping and reconstruction, respectively. Several excellent models\cite{dong2016accelerating, Kim_2016_DRCN, LapSRN_CVPR2017} are presented to improve the performance based on CNNs.

In general, the more layers the CNN model has, the better the model performance, but the deep model convergence speed becomes a critical issue during training. However, in Kim's work\cite{Kim_2016_DRCN}, named VDSR for short, the very deep convolutional network was proposed based on residual-learning\cite{He2015Deep}, which can effectively strengthen the transfer of the gradient and enhance the convergence speed. In their model, the magnitude of convolutional layers is up to 20, while the model presented in \cite{Dong-He-PAMI2016Image} only has 3 layers. Compared with Dong's work\cite{Dong-He-PAMI2016Image}, however, VDSR achieves better performance not only on image quality, but also on the running time. Recently, Lai et al.\cite{LapSRN_CVPR2017} proposed a Laplacian Pyramid Super-Resolution Network(LapSRN) based on a cascade of convolutional neural networks(CNN). The network progressively predicts the sub-band residual in a coarse-to-fine fashion and is trained with a robust Charbonnier loss function to reconstruct the high-frequency information.

Different from the previous works, generative adversarial network(GAN) is one of the most common methods\cite{David-BEGAN2017, Christian-SRGAN-CVPR2017, Yu-URGAN-ECCV2016, Goodfellow2014GAN} to adapt for SR. Due to the discriminative network, GAN-based methods can generate HR images with much sharper details than other generative models\cite{kingma2013auto, denton2015deep}. In order to reconstruct more realistic texture details with large upscaling factors, Christian et al.\cite{Christian-SRGAN-CVPR2017} proposed a deep residual network with the perceptual loss function which consists of an adversarial loss and a content loss. Specifically, the authors calculated the content loss based on high-level feature maps of VGG network\cite{Simonyan2015Very} instead of MSE(the mean squared error).

\subsection{Face image SR}

Face image SR, also called face hallucination, is an important branch of SR. Due to face inherently possesses specific spacial configuration (e.g., facial landmarks localization). So, it is very obvious that facial features and landmarks can be extracted as guidance of prior to recover HR face images. For example, Jiang et al.\cite{Jiang2016Noise-TCYB, Jiang2016SRLSP} proposed a face image SR method using smooth regression with local structure prior(SRLSP). The authors consider the relationship between the LR image patch and the hidden HR pixel information as local structure prior, which is then used to recover HR face image from the LR one. Because of the overlap patch mapping, the above method is time consuming.

However, Zhu et al.\cite{zhu2016deep} pointed out that is a chicken-and-egg problem - HR face image is better recovered by face spatial configuration, while the latter requires a higher resolution face image.  To address the problem, the authors proposed the \textit{Cascaded Bi-Network}(CBN) with alternatingly optimizing two branch networks(face hallucination and dense correspondence filed estimation). The latter branch is capable of reconstructing and synthesizing latent texture details from the LR face image.

The methods based on GAN architecture can also applied to generate HR face image from LR one. Different from aforementioned methods\cite{zhu2016deep, Jiang2016Noise-TCYB, Jiang2016SRLSP}, Yu et al. \cite{Yu-URGAN-ECCV2016} presents a discriminative generative network, without capturing any prior information, to recover HR face images with high upscaling factors($8\times$). However, there are two drawbacks with this method. One is that the face train set require frontal and approximately aligned, the other is that the generative face images are sensitive to rotations.

%%==========================end related work=======================%%

\section{Proposed method}
\label{gen_inst}

The aim of Single Image Super Resolution(SISR) is to estimate the mapping from lower-resolution input image $I^{LR}$ to high-resolution output images $I^{HR}$. Here the $I^{LR}$ downsample from corresponding $I^{HR}$ in a general way. Philip et at.'s\cite{pix2pix} research shows that conditional generative adversarial networks\cite{mirza2014conditionalGAN} are a promising approach for a variety of image-to-image translation tasks. Inspired by their works\cite{mirza2014conditionalGAN, pix2pix}, we considered $I^{LR}$ to $I^{HR}$ as a conditional transition task, namely $I^{LR}$ is the condition to generate $I^{HR}$. Furthermore, our proposed FCGAN method extends the Wasserstein distance\cite{Arjovsky2017TowardsWGAN, Arjovsky2017WGAN, David-BEGAN2017} to optimize the networks in our model.

\subsection{Model architectures}
The structure of our model is shown in figure \ref{Fig:architecture}. We adapt our generator and discriminator architecture from the U-Net\cite{ronneberger2015U-net} which is an encoder-decoder with skip connections between mirrored layers in the encoder and decoder stacks. The skip layer connections have been used in many solutions\cite{ronneberger2015U-net, huang2016Densenet, orhan2017skip, Densenetsemantic} in the filed of Deep Convolutional Neural Network(DCNN).

We design the network architecture around the following considerations. The skip connections can strengthen feature propagation and encourage feature reuse between the two connected layers. If not use skip connections, the information (taken by the previous feature map) will missing progressively when passed through a series of layers, and the convergence speed of the model will be also slow down sharply in the training phase.

The architecture of generator G: $R^{N_x} \rightarrow R^{N_y}$ is a fully convolutional neural network to generate HR image corresponding with the input LR image. $N_x = H\times W \times C$ is short for the dimensions of $x$ where $H,W,C$(for RGB image $C=3$) are height, width and colors, respectively. In order to make sure the dimensions of connection features in different layers to be the same, we implement the convolution with the kernel size of $4\times4$ in each layer and set $stride = 2$  to reduce the feature maps' dimensions. LeakyReLU activation($\alpha = 2$) is used, and pooling operation avoid to use throughout the network. The generator network G illustrated in the upper section of figure \ref{Fig:architecture} contains six downsampling convolutional layers and six upsampling convolutional layers with a decreasing/increasing factors of 2. In short, the structure of G can be simply referred to as the following pipeline: $128\times128\times3(input)\rightarrow 64\times64\times64 \rightarrow 32\times32\times128 \rightarrow 16\times16\times256 \rightarrow 8\times8\times512 \rightarrow 4\times4\times512 \rightarrow 2\times2\times512 \rightarrow 4\times4\times1024 \rightarrow 8\times8\times1024 \rightarrow 16\times16\times512 \rightarrow 32\times32\times256 \rightarrow 64\times64\times128 \rightarrow 128\times128\times3(output)$.

The architecture of discriminator D: $R^{N_{ry}} \rightarrow R^{N_{ry}}$, where $R^{N_{ry}}$, having the dimensions of ($H \times W \times 2C$), is grouped by the output(generative SR image simple) of G and corresponding real SR image sample. As showing in the bottom section of figure \ref{Fig:architecture}, the architecture of D is similar with G. There are only two crucial distinguishable points between G and D network, one is the input/output dimensions, the other is that D has only ten convolutional layers(five downsampling and upsampling layers).



\subsection{Loss function}

Typical GANs try to capture training data distribution\cite{Goodfellow2014GAN}: generator G learns the distribution $p_g$ over data $x$ to generate fake data $G(x)$, and discriminator D distinguishes the distribution of a sample whether belongs to real or fake data. Inspired by \cite{Arjovsky2017WGAN, zhao2016EBGAN}, our method attempts to match the loss distribution directly at the pixel level. Thus, in our model, we use the $L_1$ norm to measure the loss error between the generative sample $G(z)$ and the corresponding sample $x$. Motivate by David et al.\cite{David-BEGAN2017}, we adapt original GAN\cite{Goodfellow2014GAN} loss function as pixel-wise $L_1$ norm to optimize the generator and discriminator network loss function. The generator $L_1$ norm loss function as shown following equation \ref{eq:pixel_loss}.
\begin{align}\label{eq:pixel_loss}
    \mathcal{L}(I) = |I^{HR}-G(I^{LR})|
\end{align}


As the research of BEGAN\cite{David-BEGAN2017} shown, the image-wise loss distribution is approximately normal under condition of a sufficient substantial number of pixels. Thus, the objective function can further simplify to the equation \ref{eq:GD_loss}, where $x$ is real HR face sample, $z$(input of G) is the LR face sample, $y$ is the fake HR face image (output of G) generated by G with $z$, and $\mathcal{L}_D$ represents the global loss of D. In addition, in the equation \ref{eq:preD_loss}, where $\mathcal{L}_{D_r}$ represents the discriminator loss with real sample, $\mathcal{L}_{D_f}$ represents the discriminator loss with fake sample generated by G. Given the discriminator and generator parameters $\theta_D$ and $\theta_G$, which updated by minimizing the losses $\mathcal{L}_D$ and $\mathcal{L}_G$.
\begin{align}\label{eq:preD_loss}
y & = G(x;\theta_D) \\
\mathcal{L}_{D_r} & = \mathcal{L}(D(x;\theta_D)-x) \\
\mathcal{L}_{D_f} & = \mathcal{L}(D(y;\theta_D)-y) \notag \\
& = \mathcal{L}(D((G(z;\theta_G))-G(z;\theta_G));\theta_D)\\
\mathcal{L}_D & = \mathcal{L}_{D_r} - \mathcal{L}_{D_f}
\end{align}
\begin{equation}\label{eq:GD_loss}
  \begin{cases}
   \mathcal{L}_D = \mathcal{L}_{D_r} - \mathcal{L}_{D_f}, & \text{for } \theta_D \\
   \mathcal{L}_G = \mathcal{L}(G(z)-x), & \text{for } \theta_G
  \end{cases}
\end{equation}
%\begin{align}\label{eq:G_loss}
%\mathcal{L}_G = \mathcal{L}(G(z)-x)
%\end{align}

To maintain the optimization level between the generator G and discriminator D, we finally use the equilibrium algorithm\cite{David-BEGAN2017} as shown in the equation \ref{eq:Equilibrium}. If not, the parameters of generative network may be optimized in a high level, but the discriminator is still in poor level. The essential idea of the algorithm is a form of closed-loop feedback control to maintain the balance of the whole training process. We set $\gamma = 0.5$, $\lambda = 0.001$ in our experiments.
\begin{align}\label{eq:Equilibrium}
\begin{cases}
  \mathcal{L}_D = \mathcal{L}_{D_r} - k_t \mathcal{L}_{D_f} \\
  k_{t+1} = k_t + \lambda_k(\gamma \mathcal{L}_{D_r} - \mathcal{L}_G)
\end{cases}
\end{align}

Furthermore, we employ $\mathcal{M}_c$ \cite{David-BEGAN2017} (as shown in the equation \ref{eq:measure}) to measure the convergence level of our model.
\begin{align}\label{eq:measure}
\mathcal{M}_c = \mathcal{L}_{D_r} + |\gamma \mathcal{L}_{D_r}-\mathcal{L}_G|
\end{align}

These equations, while similar to those from BEGAN, have two important differences:
\begin{itemize}
  \item The input of generator, which not a random vector sample, is LR face image. We regard the input as a condition for generating HR face image. Thus, our method can control the generative face.
  \item We use $L_1$ norm as the pixel-wise loss functions of generator, as the equation \ref{eq:GD_loss} shown.
\end{itemize}


\section{Experiments}

We trained our model using Adam with the learning rate of 0.0001. After 10 iterations training with CelebA\cite{liu2015Celeba} face dataset, our model converged to its final state, which spend about 120 minutes in the machine (one NVIDIA TITAN X GPU, 12G). In order to demonstrate the performance of FCGAN, we will compare our results to the state-of-the-art methods\cite{dong2016accelerating, LapSRN_CVPR2017, pix2pix} and evaluate it qualitatively and quantitatively in the section 4.2.

\begin{figure}[h]
  \centering
  \includegraphics[width = 8.5cm]{compared-results.png}
  \caption{Comparison with the state-of-the-art methods training with CelebA dataset. (a) LR images. (b) Bicubic interpolation. (c) Philip et al.'s method\cite{pix2pix}. (d) Dong et al.'s method\cite{dong2016accelerating}. (e) Lai et al.'s method\cite{LapSRN_CVPR2017}. (f) FCGAN(ours). (g) Original HR images.}
  \label{Fig:comparison}
\end{figure}

\subsection{Setup}
\embf{Datasets.} CelebA\cite{liu2015Celeba} is a large-scale face attributes dataset with more than 200k celebrity images, each with 40 attribute annotations. The dataset covers large pose variations and background clutter. Before training our proposed model with CelebA dataset, we cropped the images and resize them to $128\times128$. We randomized the cropped images, and then used more than 180k images for training, 10k images for validation, 10k images for testing.

\embf{Set up LR datasets.} Firstly, we downsample the HR images ($128 \times 128$) to the resolution of $32\times32$ pixels (LR images). Then, we employ bicubic interpolation algorithm to generate interpolative images (named BHR, with the size of $128\times128$), and finally constructed the BHR and HR images to the input-output pairs($b_{i},h_{i}$). So, the input and output images of FCGAN are same size of $128\times128$ with three color channels.

\subsection{Experimental results and analysis}

In this section, we compare our FCGAN with currently state-of-the-art SR methods. In order to make a fair comparison, we retrain all other algorithms with the dataset CelebA. We report the qualitative results in figure \ref{Fig:comparison}, and provide the quantitative results in table \ref{tab:PSNR}. Furthermore, the figure \ref{Fig:local_details} shows the more clearly local details of the generative HR images. As can be seen from the results, our FCGAN method has significant advantages over other methods.



\begin{figure}[h]
  \centering
  \includegraphics[width = 8.5cm]{local_detail.jpg}
  \caption{Visual comparisons on local details}
  \label{Fig:local_details}
\end{figure}

\begin{table}
  \caption{Quantitative comparisons on the CelebA dataset}
    \label{tab:PSNR}
  \begin{tabular}{ccccccc}
    \toprule
  % after \\: \hline or \cline{col1-col2} \cline{col3-col4} ...
    & LR & bicubic & pix2pix & FSRCNN & LapSRN & ours \\
    \midrule
  PSNR & 29.46 & 31.25 & 30.27 & 31.92 & 32.13 & 32.42 \\
    \bottomrule
\end{tabular}
\end{table}


%The reader may be interested in an GAN-based face image SR solution on %Github\footnote{\url{https://github.com/david-gpu/srez}}.
As shown in figure \ref{Fig:gene_results}, more results generated by our FCGAN method are listed. It is worth pointing out that FCGAN can robustly generate high-quality face images ($4 \times$) regardless of facial expression, pose, illumination, occlusion(wearing glasses or hat), and other factors.

\begin{figure}[h]
  \centering
  \includegraphics[width = 9cm]{generate_results3.jpg}
  \caption{Qualitative HR face images generated by our method with $4\times$ upscaling factors}
  \label{Fig:gene_results}
\end{figure}




% An example of a floating figure using the graphicx package.
% Note that \label must occur AFTER (or within) \caption.
% For figures, \caption should occur after the \includegraphics.
% Note that IEEEtran v1.7 and later has special internal code that
% is designed to preserve the operation of \label within \caption
% even when the captionsoff option is in effect. However, because
% of issues like this, it may be the safest practice to put all your
% \label just after \caption rather than within \caption{}.
%
% Reminder: the "draftcls" or "draftclsnofoot", not "draft", class
% option should be used if it is desired that the figures are to be
% displayed while in draft mode.
%
%\begin{figure}[!t]
%\centering
%\includegraphics[width=2.5in]{myfigure}
% where an .eps filename suffix will be assumed under latex,
% and a .pdf suffix will be assumed for pdflatex; or what has been declared
% via \DeclareGraphicsExtensions.
%\caption{Simulation results for the network.}
%\label{fig_sim}
%\end{figure}

% Note that the IEEE typically puts floats only at the top, even when this
% results in a large percentage of a column being occupied by floats.


% An example of a double column floating figure using two subfigures.
% (The subfig.sty package must be loaded for this to work.)
% The subfigure \label commands are set within each subfloat command,
% and the \label for the overall figure must come after \caption.
% \hfil is used as a separator to get equal spacing.
% Watch out that the combined width of all the subfigures on a
% line do not exceed the text width or a line break will occur.
%
%\begin{figure*}[!t]
%\centering
%\subfloat[Case I]{\includegraphics[width=2.5in]{box}%
%\label{fig_first_case}}
%\hfil
%\subfloat[Case II]{\includegraphics[width=2.5in]{box}%
%\label{fig_second_case}}
%\caption{Simulation results for the network.}
%\label{fig_sim}
%\end{figure*}
%
% Note that often IEEE papers with subfigures do not employ subfigure
% captions (using the optional argument to \subfloat[]), but instead will
% reference/describe all of them (a), (b), etc., within the main caption.
% Be aware that for subfig.sty to generate the (a), (b), etc., subfigure
% labels, the optional argument to \subfloat must be present. If a
% subcaption is not desired, just leave its contents blank,
% e.g., \subfloat[].


% An example of a floating table. Note that, for IEEE style tables, the
% \caption command should come BEFORE the table and, given that table
% captions serve much like titles, are usually capitalized except for words
% such as a, an, and, as, at, but, by, for, in, nor, of, on, or, the, to
% and up, which are usually not capitalized unless they are the first or
% last word of the caption. Table text will default to \footnotesize as
% the IEEE normally uses this smaller font for tables.
% The \label must come after \caption as always.
%
%\begin{table}[!t]
%% increase table row spacing, adjust to taste
%\renewcommand{\arraystretch}{1.3}
% if using array.sty, it might be a good idea to tweak the value of
% \extrarowheight as needed to properly center the text within the cells
%\caption{An Example of a Table}
%\label{table_example}
%\centering
%% Some packages, such as MDW tools, offer better commands for making tables
%% than the plain LaTeX2e tabular which is used here.
%\begin{tabular}{|c||c|}
%\hline
%One & Two\\
%\hline
%Three & Four\\
%\hline
%\end{tabular}
%\end{table}


% Note that the IEEE does not put floats in the very first column
% - or typically anywhere on the first page for that matter. Also,
% in-text middle ("here") positioning is typically not used, but it
% is allowed and encouraged for Computer Society conferences (but
% not Computer Society journals). Most IEEE journals/conferences use
% top floats exclusively.
% Note that, LaTeX2e, unlike IEEE journals/conferences, places
% footnotes above bottom floats. This can be corrected via the
% \fnbelowfloat command of the stfloats package.


\section{Conclusion and future work}
In this paper, we have proposed a novel SR method ($4\times$ upscaling factors) to generate a HR face image from LR one, namely \textit{Face Conditional Generative Adversarial Network} (FCGAN). In this model, the LR image, instead of random noise, is considered as a controller to generate a HR image. Our FCGAN is an end-to-end framework, without any pre/post-processing (e.g., face alignment, extracting facial structure prior information). Furthermore, it is a robustly model, the generative image is not sensitive to facial expression, pose, illumination, occlusion (wearing glasses or hat), and so on. For the generator and discriminator networks, the skip-layer connection technique is utilized for enhancing the convergence speed in the training phase. Thus, our model has great advantages on the training time over other SR models based on CNN.
%收敛速度快。
%L1 norm

However, there are several problems that worth to further investigate in the future. We note that the input image size of recent FCGAN model is same as the generative HR image ($128\times128$). In the future research, we will design an advanced model that can directly generate HR face image (e.g., $128\times128$) from the small size one (e.g., $32\times32$). In addition, we only show the excellent performance on face image SR task in this work, and it is worth to extend our proposed framework for the task of generic image SR.

% conference papers do not normally have an appendix
% 未来研究方向:1) input size,32->128 2) upscaling factors, 16->128(8X) 3) extend to generic image SR.
%


% use section* for acknowledgment
\ifCLASSOPTIONcompsoc
  % The Computer Society usually uses the plural form
  \section*{Acknowledgments}
\else
  % regular IEEE prefers the singular form
  \section*{Acknowledgment}
\fi

This work is supported by the National Natural Science Foundation of China under Grant 73003802 and the Major International (Regional) Joint Research Project of China under Grant 73045702.

% trigger a \newpage just before the given reference
% number - used to balance the columns on the last page
% adjust value as needed - may need to be readjusted if
% the document is modified later
%\IEEEtriggeratref{8}
% The "triggered" command can be changed if desired:
%\IEEEtriggercmd{\enlargethispage{-5in}}

% references section

% can use a bibliography generated by BibTeX as a .bbl file
% BibTeX documentation can be easily obtained at:
% http://mirror.ctan.org/biblio/bibtex/contrib/doc/
% The IEEEtran BibTeX style support page is at:
% http://www.michaelshell.org/tex/ieeetran/bibtex/
%\bibliographystyle{IEEEtran}
% argument is your BibTeX string definitions and bibliography database(s)
%\bibliography{IEEEabrv,../bib/paper}
%
% <OR> manually copy in the resultant .bbl file
% set second argument of \begin to the number of references
% (used to reserve space for the reference number labels box)
%\begin{thebibliography}{1}
%
%\bibitem{IEEEhowto:kopka}
%H.~Kopka and P.~W. Daly, \emph{A Guide to \LaTeX}, 3rd~ed.\hskip 1em plus
%  0.5em minus 0.4em\relax Harlow, England: Addison-Wesley, 1999.
%
%\end{thebibliography}


\bibliography{refer}
%\bibliography{refer}

% that's all folks
\end{document}


