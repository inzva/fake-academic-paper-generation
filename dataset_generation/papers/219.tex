\documentclass[10pt,twocolumn,letterpaper]{article}

\usepackage{cvpr}
\usepackage{times}
\usepackage{epsfig}
\usepackage{graphicx}
\usepackage{amsmath}
\usepackage{amssymb}
\usepackage{mathrsfs}
\usepackage{subfigure}

\long\def\symbolfootnote[#1]#2{\begingroup
\def\thefootnote{\fnsymbol{footnote}}\footnote[#1]{#2}\endgroup}

% Include other packages here, before hyperref.

% If you comment hyperref and then uncomment it, you should delete
% egpaper.aux before re-running latex.  (Or just hit 'q' on the first latex
% run, let it finish, and you should be clear).
\usepackage[pagebackref=true,breaklinks=true,letterpaper=true,colorlinks,bookmarks=false]{hyperref}

 \cvprfinalcopy % *** Uncomment this line for the final submission

\def\cvprPaperID{1679} % *** Enter the CVPR Paper ID here
\def\httilde{\mbox{\tt\raisebox{-.5ex}{\symbol{126}}}}

% Pages are numbered in submission mode, and unnumbered in camera-ready
\ifcvprfinal\pagestyle{empty}\fi
\begin{document}

\newcommand{\authorfont}{\fontsize{11pt}{\baselineskip}\selectfont}

%%%%%%%%% TITLE
\title{Deep Structured Scene Parsing by Learning with Image Descriptions}

\author{ \authorfont{Liang Lin$^1$, Guangrun Wang$^{1}$, Rui Zhang$^{1}$, Ruimao Zhang$^{1}$, Xiaodan Liang$^1$, Wangmeng Zuo$^2$}\\
{\small $^1$School of Data and Computer Science, Sun Yat-sen University, Guangzhou, China} \\
{\small $^2$School of Computer Science and Technology, Harbin Institute of Technology, China}\\
{\tt\small linliang@ieee.org; r.m.zhang1989@gmail.com; cswmzuo@gmail.com.}
% For a paper whose authors are all at the same institution,
% omit the following lines up until the closing ``}''.
% Additional authors and addresses can be added with ``\and'',
% just like the second author.
% To save space, use either the email address or home page, not both
%\and
%Second Author\\
%Institution2\\
%First line of institution2 address\\
%{\tt\small secondauthor@i2.org}
}

\maketitle
%\thispagestyle{empty}

%%%%%%%%% ABSTRACT
\begin{abstract}
This paper addresses a fundamental problem of scene understanding: How to parse the scene image into a structured configuration (i.e., a semantic object hierarchy with object interaction relations) that finely accords with human perception. We propose a deep architecture consisting of two networks: i) a convolutional neural network (CNN) extracting the image representation for pixelwise object labeling and ii) a recursive neural network (RNN) discovering the hierarchical object structure and the inter-object relations. Rather than relying on elaborative user annotations (e.g., manually labeling semantic maps and relations), we train our deep model in a weakly-supervised manner by leveraging the descriptive sentences of the training images. Specifically, we decompose each sentence into a semantic tree consisting of nouns and verb phrases, and facilitate these trees discovering the configurations of the training images. Once these scene configurations are determined, then the parameters of both the CNN and RNN are updated accordingly by back propagation. The entire model training is accomplished through an Expectation-Maximization method. Extensive experiments suggest that our model is capable of producing meaningful and structured scene configurations and achieving more favorable scene labeling performance on PASCAL VOC 2012 over other state-of-the-art weakly-supervised methods.

\end{abstract}

%%%%%%%%% BODY TEXT
%\vspace{-4mm}
\section{Introduction}

Scene understanding started with the goal of creating systems that can infer meaningful configurations (e.g., parts, objects and their compositions with relations) from imagery like humans~\cite{DBLP:ImageParsing-Attribute}. In computer vision research, significant progresses have been made in semantic scene labeling / segmentation (i.e., assigning the label for each pixel of the scene image)~\cite{DBLP:SemanSeg-1}\cite{DBLP:SemanSeg-3}\cite{DBLP:FCnetwork}\cite{DBLP:RecursiveContext}. However, the problem of structured scene parsing (i.e., producing meaningful scene configurations) remains a challenge due to the following difficulties.

\begin{itemize}

\item The representations of nested hierarchical structure in scene images are often ambiguous, \eg, a configuration may have more than one way of parsing. Conducting these parsing results to finely accord with human perception is an interesting yet fundamental problem.



\item Training a scene parsing model usually relies on very expensive manual annotations, \eg, including semantic maps and structured configurations.

\end{itemize}

\begin{figure}[t]
\centering
\includegraphics[width=3.3in]{fig_application}
\caption{An illustration of our structured scene parsing. An input scene image is automatically parsed into a hierarchical configuration that comprises hierarchical semantic objects (black labels) and the interaction relations (red labels) of objects.}
\label{fig:application}
\vspace{-4mm}
\end{figure}

To address these above issues, we develop a novel deep neural network architecture that automatically parses an input scene into a structured and meaningful configuration. Fig.~\ref{fig:application} shows an illustration of our structured scene parsing, where our model identifies salient semantic objects in the scene and generates the hierarchical scene structure with the interaction relations among objects. Our model is inspired by the effectiveness of two widely successful deep learning techniques: convolutional neural networks (CNNs)~\cite{DBLP:AlexNet}\cite{DBLP:FCnetwork} and recursive neural networks (RNNs)~\cite{DBLP:Recursive_Socher}. The former category of models is widely applied for generating powerful feature representations in various vision tasks such as image classification and object recognition. Meanwhile, the RNN models (such as \cite{DBLP:Recursive_Socher}\cite{DBLP:RecursiveContext}\cite{DBLP:RecursiveContext2}) have demonstrated as an effective class of models for predicting hierarchical and compositional structures in image and natural language understanding~\cite{DBLP:RNN-NLP}. One important property of RNNs is the ability to recursively learn the representations in a semantically and structurally coherent way. In our deep CNN-RNN architecture, the CNN and RNN models are collaboratively integrated for accomplishing the scene parsing from complementary aspects. We utilize the CNN to layerwise extract features from the input scene image and generate the representations of semantic objects. Then, the RNN is sequentially stacked based on the CNN feature representations, generating the structured configuration of the scene.


On the other hand, to avoid relying on the elaborative annotations, we propose to train our CNN-RNN model by leveraging the image descriptions. Our approach is partially motivated but different with the recently proposed methods for image-sentence embedding~\cite{DBLP:VisualAlign-feifei}\cite{DBLP:TellMeShowYou}. In particular, we distill knowledge from the sentence descriptions for discovering scene structural configurations.

In the initial stage, we decompose each sentence into a normalized semantic tree consisting of nouns and verb phrases by using a standard parser~\cite{DBLP:conf/acl/SocherBMN13} and the WordNet\cite{wordnet}. Afterward, based on these semantic trees and their associated scene images, we train our model by developing an Expectation-Maximization method. Specifically, the semantic tree facilitates discovering the latent scene configuration in the two following aspects. i) The entities (\ie, nouns) determine the object category labels existing in the scene, and ii) the relations (\ie, verb phrases) over the entities assist to produce the scene hierarchy and object interactions. The two proportions of knowledge are incorporated into our learning objective together with the CNN and the RNN, respectively. Therefore, once the scene configuration is fixed, the parameters of the two neural networks are updated accordingly by the back propagation.


The main contributions of our work are summarized as follows. i) We present a novel CNN-RNN framework for generating meaningful and hierarchical scene representations, which gains a deeper understanding of the objects in the scene compared to traditional scene labeling. The integration of CNN and RNN models is general to be extended to other high-level computer vision tasks. ii) We present a EM-type training method by leveraging text descriptions that associate with the training images. This method is cost-effective yet beneficial to introducing rich contexts and semantics. iii) Our extensive experiments on PASCAL VOC 2012 demonstrate that the parsed scene representations are useful for scene understanding and our generated semantic segmentations are more favorable than those by other weakly-supervised scene labeling methods.

% The rest of the paper is organized as follows. Section 2 presents a brief review of related work. Section 3 introduces our deep CNN-RNN architecture, followed by a discussion of the model training method in Section 4. The experimental results, comparisons and component analysis are presented in Section 5. Section 6 concludes the paper.


%------------------------------------------------------------------------
\section{Related Work}

Scene understanding is arguably considered as the most fundamental problem in computer vision, which actually involves several tasks of different level. In current research, a myriad of different methods focus on what general scene type the image shows (classification)~\cite{DBLP:VisualAttr}\cite{DBLP:Multi-Class}\cite{DBLP:fine-grained}, what objects and their locations are in a scene (semantic labeling or segmentation)~\cite{DBLP:Seg-03}\cite{DBLP:Seg-09}\cite{DBLP:Seg-15}\cite{tighe2014scene}. These methods, however, ignore or over-simplified the compositional object representations and would fail to gain a deeper scene understanding.

% In recent years, scene understanding attracts much attention in computer vision area. The purpose is concentrated on dealing with three levels tasks: to understand i) what objects are in the scene (scene classification problem), iiwhere the objects are located (scene labeling problem) and iii) how is the interaction between the objects (scene paring problem). In literature, a series of studies have been proposed for object classification~\cite{DBLP:VisualAttr}\cite{DBLP:Multi-Class}\cite{DBLP:fine-grained} and semantic labeling~\cite{DBLP:Seg-03}\cite{DBLP:Seg-09}\cite{DBLP:Seg-15}. However, these approaches lack a deeper understanding about meaningful and structured scene configurations among objects.

Meanwhile, as a higher-level task, structured scene parsing has also attracted much attention. A pioneer work was proposed by Tu et al.,~\cite{tu2005image}, in which they mainly focused on faces and texture patterns by a Bayesian inference framework. In \cite{DBLP:ImageParsing-Attribute}, Han et al., proposed to hierarchically parse the indoor scene images by developing a generative grammar model. A hierarchical model was proposed in \cite{zhu2012recursive} to represent the image recursively by contextualized templates at multiple scales, and the rapid inference was realized based on dynamic programming. Ahuja et al.,~\cite{ahuja2008connected} developed a connected segmentation tree for object and scene parsing. Some other related works~\cite{silberman2012indoor}\cite{gupta2013perceptual} investigated the approaches for RGB-D scene understanding, achieving impressive results.


% The target of scene parsing focus on parsing the input scene into semantic objects with their interaction relations in hierarchical. Some approaches such as~\cite{DBLP:ImageParsing-Attribute} explore hierarchical decomposition structure of the scene image. In these works, an attribute graph grammar is employed as the generative representation to parse indoor images in the process of maximizing a Bayesian posterior probability. The extension study also explored the more complex outdoor environment in~\cite{DBLP:3Dparsing-liu}. In~\cite{DBLP:ImageParsing-zhu}, Zhao et al. proposed a generative stochastic scene grammar (SSG) to generate the scene structure by an effective MCMC based inference algorithm. Although these statistic based methods considered the reconfigurable structures of scene image, the complex inference algorithm often limits their scalability.

With the resurgence of neural network models, the performances of scene understanding have been improved substantially. The representative works, the fully convolutional network (FCN)~\cite{DBLP:FCnetwork} and its extensions~\cite{DBLP:CNN-CRF}, demonstrate effectiveness in pixel-wise scene labeling. A recurrent neural network model was proposed in \cite{DBLP:CRF-RNN}, which improves the segmentation performance by incorporating the mean-field approximate inference, and similar idea was also explored in \cite{DBLP:CNN-MRF}. For the problem of structured scene parsing, recursive neural networks (RNNs) were studied in \cite{DBLP:Recursive_Socher}\cite{DBLP:RecursiveContext2}. For example, Socher et al.~\cite{DBLP:Recursive_Socher} proposed to predict hierarchical scene structures by using a max-margin RNN model. The differences between these existing RNN-based parsing models and our model are two-fold. First, they mainly focused on parsing only the semantic entities (\eg, buildings, bikes, trees) and the scene configurations generated by ours include not only the objects but also the interaction relations of objects. Second, we incorporate convolutional feature learning into our deep model for joint optimization.


% With the success of convolutional neural network (CNN), the performance of object classification has achieved great improvement~~\cite{DBLP:AlexNet}\cite{ILSVRC15}. Motivated by this, the fully convolutional network (FCN)~\cite{DBLP:FCnetwork} is proposed for semantic labeling by transfer learning and spatially dense prediction. In~\cite{DBLP:CNN-CRF}, the model combines the output of FCN with a fully connected conditional random field (CRF) to overcome the poor localization property of deep networks. Further more, the recurrent neural network is introduced in~\cite{DBLP:CRF-RNN} to formulate the mean-field approximate inference for CRF with Gaussian pairwise potentials. The similar idea is also applied in~\cite{DBLP:CNN-MRF} to incorporate the markov random field (MRF) into deep networks by carefully devising kernels in the last layer. These methods achieve state-of-the-arts performance on semantic labeling, but all of them ignore the potential deep structure in the scene image.


\begin{figure*}[t!]
\centering
\includegraphics[width=6.8 in]{fig_inference_new}
\caption{A glance into our proposed CNN-RNN architecture for structured scene parsing. The CNN takes the image as input and produces the pixel-wise semantic score map. Then the pixels with the same label are grouped into a semantic object, and we can obtain the feature representations (i.e., $v_1, v_2,..v_k$) of objects. Furthermore, the RNN take these feature representations of objects as input to construct the parsing tree, where $v_i$ is mapped into a semantic representation $x_i$.}
\label{fig:inference}
\vspace{-2mm}
\end{figure*}

Most of the existing scene labeling / parsing models are studied in the context of supervised learning, and they rely on expensive annotations. To overcome this issue, one can develop alternative methods that train the models from weakly annotated training data, e.g., image-level tags and contexts \cite{DBLP:Weakly2}\cite{DBLP:Weakly-MultiInstance}\cite{DBLP:WeaklySegmentation}. Among these methods, one inspiring us is \cite{DBLP:WeaklySegmentation}, which adopts an EM learning algorithm for training the model with image-level semantic labels. This algorithm alternates between predicting the latent pixel labels subject to the weak annotation constraints and optimizing the neural network parameters.



% Recently, some deep recursive neural network (RNN) based approaches~\cite{DBLP:Recursive_Socher}\cite{DBLP:RecursiveContext}\cite{DBLP:RecursiveContext2}were explored for semantic scene parsing. In~\cite{DBLP:Recursive_Socher}, a max-margin structure prediction architecture was introduced to predict hierarchical tree structure for scene images. It showed that constructing such structure could effectively promote in merging superpixels into semantic regions. Inspired by this work, Sharma et al. proposed the deep recursive context propagation network (RCPN) in~\cite{DBLP:RecursiveContext} and~\cite{DBLP:RecursiveContext2}. This deep feed-forward neural network utilizes the contextual information from the entire image to update the feature representation of each superpixel to achieve the better classification performance. There are two differences between previous RNN based methods and our approach : i) previous RNN based methods are more similar to the studies in ~\cite{DBLP:RecurrentNNforlabeling}\cite{DBLP:SemSeg-Context}\cite{DBLP:HierFea-Scelab}\cite{DBLP:Inference-SemSeg}, which are concentrate on classifying each pixel (or superpixel) to a semantic entities (building, bike, tree, airplane etc.), but ignore the interaction relation among them. ii) due to the lack of consistency constraint, the predicted structures of these RNN models are not always unique.


% All the works mentioned above are in the supervised manner. However, labeling data for semantic labeling or parsing is expensive compared with the classification problem. Thus a serious of weakly supervised approaches have been proposed in the past years by applying image-level annotation~\cite{DBLP:Weakly2}\cite{DBLP:TellMeShowYou}\cite{DBLP:ImageToPixelLabel}\cite{DBLP:Weakly-MultiInstance}\cite{DBLP:WeaklySegmentation}\cite{DBLP:LabeltoRegion-liu}. Among them, our work is most relate to \cite{DBLP:WeaklySegmentation}, which treats the pixel label as the latent variable and proposes an EM based learning algorithm to inference latent variables and optimize parameters of the network by taking the weak labels into consideration. Different from this method, our model applies the sentence description to label the salient semantic object in the image. By employing such transfer knowledge, our method seems more reasonable and flexible for real applications.


%At last, some recent works about image-sentence embedding and mapping~\cite{DBLP:I2T}\cite{DBLP:ConnectModa-feifei}\cite{DBLP:VisualAlign-feifei}\cite{DBLP:CrossModalRetrieval} are also related to this paper. The key idea of these studies is to discover the inter-modal correspondences between language and visual data and align components in the two modalities. Our work is motivated by these models, but we just use sentence description to avoid the elaborative annotation and discover the scene configuration automatically.

%\textbf{Recursive Neural Network.}


\section{CNN-RNN Architecture}
\label{CNN-RNN Architecture}

Structured scene parsing aims to infer the following three forms of outputs from an image: i) the location of semantic entities, ii) interaction relations and iii) the hierarchical configuration among the semantic entities. To this end, we propose a novel deep architecture by integrating the convolutional neural network (CNN) and recursive neural network (RNN). In our CNN-RNN architecture, the CNN model is introduced to perform semantic segmentation by assigning an entity label (i.e. object category) to each pixel, and the RNN model is introduced to discover hierarchical structure and interaction relations among entities.

Fig.~\ref{fig:inference} illustrates the the proposed CNN-RNN architecture for structured scene parsing. First, the input image is directly fed into our revised VGG-16 network~\cite{vggnet} to produce a score map for each entity category. Based on the softmax normalization of the score maps, each pixel is labeled with an entity category. We further group the adjacent pixels with the same label into an object, and obtain the feature representations of objects. By feeding these feature representations of entities to the RNN, a greedy aggregation procedure is implemented for  constructing the parsing tree. In each recursive iteration, two input objects (denoted by the child nodes) are merged into a higher-level object (denoted the parent node). The finally generated root note represents the whole scene. Different from the RNN architecture in~\cite{DBLP:Recursive_Socher}\cite{DBLP:RecursiveContext2}, our model predicts the relation between these two nodes when they are combined into a higher-level node.

In the following, we discuss the CNN and RNN models in details.




% {The complete flow diagram of CNN-RNN architecture for structured scene parsing. The input image is directly fed into CNN to product score map for each semantic category and feature representation of each pixel. Then the model applies score maps to classify the pixels, and groups the ones with same labels as a semantic entity. According to the pixel-level features, the feature representation $v$ of each semantic entity is calculated and put into RNN to predict the tree structure and relations between entities.}

%The objective of this paper is to parse the content of a natural image as a set of component and represent the parsing result as a tree structure. The proposed method targets on predicting three kinds of outputs: image semantic segmentation, relationship of semantic regions and structure of the paring tree. First, we represent the input image $I$ as a set pixels. For $K$ different semantic labels, we denote their corresponding semantic region as  $R=\{r_k\}_{k=1}^K$, each of which is associated with a set of pixels. We introduce $C=\{c_j\}_{j=1}^M$ to indicate the labels of pixels in the image $I$, where $c_j\in \{1,...,K\}$ and $M$ is the number of pixels in image $I$. The purpose of semantic segmentation is to group pixels into semantic region according to their labels, which are predicted from the convolution neural network. Second, the feature vectors of two related regions are denoted by $x_i$ and $x_j$, respectively. We apply $b_{ij}$ to indicate the contextual information of semantic region $i$ and $j$. The relationship of two semantic regions is predicted by classifier $F(\{x_i,x_j,b_{ij}\})$. At last, the recursive neural network outputs the aggregation order of each semantic region. And the tree structure of the parsing result of the input image $I$ is generated according to these aggregation orders.



%In this paper, we train the model of three kinds of outputs in a weakly supervised manner with an end-to-end framework, which jointly optimizes parameters of convolutional neural network (CNN) and recursive neural network (RNN). The former one predicts the semantic segmentation result and the latter one predicts the relationship of semantic regions and output the tree structure of the semantic parsing result. In order to learn parameters in the neural networks, we introduce the text description $T$ to describe the content of image $I$. Such representation of natural language provides all the information for the training process, \textit{e.g.} some nouns indicate the labels of semantic regions, some verbs or prepositions indicate the categories of relationships between regions and the text parsing tree provides the structure constraint when generating the image parsing tree.






\subsection{CNN Model}
\label{sub:cnn_model}
% The target of CNN model is to assign one semantic label for each pixel location in the image, and cluster the pixels with the same label as a semantic entity. For $K$ different semantic labels, we denote their corresponding semantic entities as  $E=\{e_k\}_{k=1}^K$, each of which is associated with a set of pixels. We introduce $C=\{c_j\}_{j=1}^M$ to indicate the labels of pixels in the image $I$, where $c_j\in \{1,...,K\}$ and $M$ is the number of pixels in image $I$. During the inference phase, the network produces one score $s_j^k$ associated with class $k$ for each pixel $j$. Thus we can obtain $K+1$ feature maps in the last layer of neural network. The first $K$ maps are associated with semantic labels and the last one indicates the background. In order to convert the score $s_j^k$ to a classification score. we calculate the class score of each pixel applying a softmax regression:
% \begin{equation}\label{eq_softmax}
% \sigma(s_j^t) = \frac{\exp(s_j^t)} {\sum_{k=1}^K \exp(s_j^k)}
% \end{equation}
% where $\sigma(s_j^t)$ indicates the probability of $j$-th pixel belonging to $t$-th semantic category, and $\sum_{t=1}^K \sigma(s_j^t)=1$. Thus the class label of $j$-th pixel can be predicted by:
% \begin{equation}\label{eq_prediction}
% c_j = \arg\max_t \: \sigma(s_j^t)
% \end{equation}
% After obtaining the semantic label of each pixel, we group the adjacent pixels with the same label into the semantic entity. In order to calculate the feature representation of each semantic region, a convex approximation of the \textit{max} function, called \textit{Log-Sum-Exp}(LSE)~\cite{}, is applied:


The CNN model is designed to accomplish two tasks: semantic labeling and generating feature representations for entities. For semantic labeling, we adopt the fully convolutional network with parameters $W_C$ to yield $K+1$ score maps $\{{s}^0, ..., {s}^k, {s}^K\}$, corresponding to one extra background category and $K$ object categories. The score $s_j^k$ is further normalized using softmax to obtain the corresponding classification score:
\begin{equation}\label{eq_softmax}
\sigma(s_j^t) = \frac{\exp(s_j^t)} {\sum_{k=1}^K \exp(s_j^k)}
\end{equation}
where $\sigma(s_j^t)$ denotes the probability of $j$-th pixel belonging to $t$-th object category with $\sum_{t=1}^K \sigma(s_j^t)=1$. $C=\{c_j\}_{j=1}^M$ denotes the labels of pixels in the image $I$, where $c_j\in \{1,...,K\}$ and $M$ is the number of pixels of image $I$. With $\sigma(s_j^t)$, the label of the $j$-th pixel can be predicted by:
\begin{equation}\label{eq_prediction}
c_j = \arg\max_t \: \sigma(s_j^t)
\end{equation}
%For $K$ different semantic categories, we denote their corresponding semantic entities as  $E=\{e_k\}_{k=1}^K$, each of which is associated with a set of pixels.

For generating feature representation for each entity category, we group the adjacent pixels with the same label into a semantic entity category.

Considering that the pixel numbers vary with the semantic entity categories, in order to obtain feature representation with fixed length for any entity category, we use \textit{Log-Sum-Exp}(LSE)~\cite{boyd2004convex}, a convex approximation of the \textit{max} function, to fuse the features of pixels
\begin{equation}\label{eq_feature}
v_k = \frac{1}{\pi}\log \left[\frac{1}{Q_k}\sum_{c_j = k} \exp (\pi \bar{v}_j) \right]
\end{equation}
where $v_k$ denotes the feature representation of the $k$-th entity category, $\bar{v}_j$ denotes the feature representation of the $j$-th pixel by concatenating all feature maps at the layer before softmax at position $j$ into a vector, $Q_k$ is the total number of pixels of the $k$-th object category, and $\pi$ is a hyper-parameter to control smootheness. With higher value of $\pi$, the function tend to preserve the max value for each dimension in the feature, while with lower value the function behaves like a averaging function.


%%%%%%%%%%%%%%%%%%%%%%%

\iffalse
\fi

%%%%%%%%%%%%%%%%%%%%%%%%









%\textbf{Pixel Grouping and Region Smoothing.} After obtaining the semantic label of each pixel, we group the adjacent pixels with the same label into the semantic region. In addition, considering the prediction error of the softmax classification, we apply the sliding window refinement to further increase the regional consistency. Specifically, we employ the sliding window with the size $a\times a$ (\textit{i.e.} $5\times 5$ in practice) to predict the new label of each position. For each pixel, we count the number of its neighboring pixels of each category in the sliding window. If the number is larger the threshold for a certain category $l_k$, we then replace the semantic label of central pixel with $l_k$, else the central pixel is set as background. Actually, since our region smoothing only calculates the amount of pixels belonging to each semantic label without any inference, our method is very simple and fast comparing to the CRF model define in~\cite{,}.





\subsection{RNN Model}
\label{sub:rnn_model}
%This section contains an explanation of semantic similarity between corresponding nodes from the text parsing tree and the image parsing tree. And how to keep the structural consistency of two parsing trees.

%\subsubsection{Overview}

%Since the recursive structure is commonly found in natural images, recursive neural network (RNN) is first introduced to discover the structure of different parts of the image in~\cite{}. In~\cite{} and \cite{}, such powerful deep architecture is also applied to parse the input image into the semantic regions. Different from these previous works, we use recursive neural network to formulate the construction of image parsing tree. Specifically, such process can be divided into relationship prediction and structure preservation. The former one focus on predicting the relationship of two semantic regions in image. At the same time, the prediction result should match the corresponding verb in the sentence descriptor. The latter is concentrated on preserving the tree structure of the text in the image parsing tree. The input of the recursive neural network is the feature representation of each segmented region in  Eq.(\ref{eq_feature}). Then binary parse trees is constructed using the relevant information (corresponding to the verb in the text parsing tree) between semantic regions. The leaf nodes are associated with the initial regions and well-regulated merger of two regions builds the internal nodes up to the root node, which corresponds to the entire image. Five different kinds of modules (\textit{semantic mapper}, \textit{combiner}, \textit{interpreter} and \textit{categorizer}) are proposed to generate the final tree representation. The detailed function of each module will be introcuted in the next subsection.


\begin{figure}[t]
\centering
\includegraphics[width= 2.4in]{fig_rnn}

 \caption{An illustrate of recursive neural network in our CNN-RNN architecture. This network calculates the score for merging decision and predicts the relation category of two merged regions.}
\label{fig:RNN}
\end{figure}


% Since the recursive structure is commonly found in natural images, recursive neural network (RNN) is first introduced to discover the structure of different parts of the image in~\cite{}. In~\cite{} and \cite{}, such powerful deep architecture is also applied to parse the input image into the semantic regions. Different from these previous works,

With the feature representations of object categories produced by CNN, the RNN model is designed to generate the image parsing tree for predicting object interaction relations and hierarchical scene structure. The RNN model consists of four sub-networks: (\textit{semantic mapper}, \textit{combiner}, \textit{categorizer} and \textit{scorer}). Therefore, the parameters of the RNN also includes four parts, denoted as $W_R=\{W_{sem},W_{com},W_{cat},W_{score}\}$.

Following~\cite{DBLP:Recursive_Socher} and~\cite{DBLP:RecursiveContext2}, object feature $v_k$ produced by CNN is first mapped onto a semantic space by the \textbf{Semantic mapper}, which is a one-layer fully-connected network.
\begin{equation}\label{eq_sem}
x_k = F_{sem}(v_k;W_{sem})
\end{equation}
where $x_k$ is the mapped feature, $F_{sem}$ is the network transformation and $W_{sem}$ is the network parameter.

The features of two child nodes are fed to the \textbf{Combiner} and generate their parent node feature.
\begin{equation}\label{eq_com}
x_{kl} = F_{com}([x_k,x_l];W_{com})
\end{equation}
where $x_k$ and $x_l$ indicate the two child features and $x_{kl}$ denotes their parent feature in the parsing tree. $F_{com}$ is the network transformation and $W_{com}$ is the corresponding parameter. Parent node feature encode semantic information of the combination of its two child nodes, as well as the structural information of this specific merging operation. The parent node feature has the same dimensionality as the child node feature, allowing the procedure can be applied recursively and eventually the root feature can be used to represent the whole image.

When two nodes are merged into a parent node, the \textbf{Categorizer} sub-network determines the relation of these two nodes. Categorizer is a softmax classifier that takes parent node feature $x_{kl}$ as input, and predict the relation label $y_{kl}$,
\begin{equation}\label{eq_rel_category}
y_{kl} = softmax(F_{cat}(x_{kl};W_{cat}))
\end{equation}
where $y_{kl}$ is the predicted relation probability vector, $F_{cat}$ denotes the network transformation and $W_{cat}$ denotes the network parameter.

The \textbf{Scorer} sub-network measures the confidence of a merging operation between two nodes. It takes the parent node feature $x_{kl}$ as input and outputs a real value $h_{kl}$.
\vspace{-1mm}
\begin{equation}\label{eq_node_score}
h_{kl} = F_{score}(x_{kl};W_{score})
\end{equation}
where $F_{score}$ denotes the network transformation and $W_{score}$ denotes the network parameter. The merging score $q_{kl}$ of node $\{kl\}$ is computed as $q_{kl} = \frac{1}{1+exp(h_{kl})}$.

Merging score is used to optimize the structure discovery in training, as described in Sect. \ref{sub:rnn_loss}.

Similar to \cite{DBLP:Recursive_Socher}, we use the RNN model to construct the parsing tree with a greedy algorithm. The procedure begins with a initial set of leaf nodes. In each iteration, the algorithm enumerates all possible merging pairs and computes merging scores for each. The algorithm chooses the pair with highest score to merge, replacing the pair of nodes with their parent node. The algorithm iterates until there is only one root node left.


%%%%%%%%%%%%%%%%%%%%%%%%%%%%%%%%
% \iffalse
% \textbf{Interpreter} is the neural network that interprets the relationship of two node in the parsing tree. Intuitively, the interpreter network attempts to integrate the feature of two nodes and their contextual information such that the output of the network represent the interaction relation of two entities,
% \begin{equation}\label{eq_int}
% y_{kl} = F_{int}([x_{kl},b_{kl}];W_{int})
% \end{equation}
% where $y_{kl}$ is the relationship feature of $x_{kl}$. $F_{int}$ and $W_{int}$ indicate the network and layer weights respectively. $b_{kl}$ denotes the contextual information as follows,
% \begin{equation}\label{eq_contextual}
% b = [b^{ang},b^{dis},b^{scal}]
% \end{equation}
% where $b^{ang}$ and $b^{dis}$ reflect the spatial relation between two semantic entities, while $b^{scal}$ is employed to imply scale relation of semantic entities. Specifically, $b^{ang}$ denotes the cosine value of the angle between the center of two semantic entities. $b^{dis}$ indicates the distance of two centers. $b^{scal}$ is the size rate of such two entities. Note that the similar definition is also proposed in~\cite{}.
% \fi
%%%%%%%%%%%%%%%%%%%%%%%%%%%%%%%%%%%%%%




%%%%%%%%%%%%%%%%%%%%%%
% \iffalse
% \fi
%%%%%%%%%%%%%%%%%%%%%%%%%%%%%%




%\subsubsection{RNN for Structure Preservation}

%In our image parsing architecture, the goal is to lean a transformation $I\rightarrow \mathcal{P}_I$ according to the text parsing tree $T$, where $ \mathcal{P}_I$ is the only valid recursive parse tree for image $I$. In our model, the image parsing tree is valid if the sequence of two regions merged into a super one is along with the merge occur between their corresponding noun words in the text parsing tree. We define an sequence of correct aggregate operations as $\mathcal{A}(I)=\{a_1,...,a_{P_T}\}$. $P_T$ is the total number of aggregate operation in the text parsing tree $T$. This definition shows that the number of region aggregation in the image is consistent with the number of noun word aggregation in the text descriptor.






%We denote the set of all possible trees (\textit{i.e.} the recursive parsing tree and its subtrees ) that can be constructed from an input image $I$ as $\mathcal{A}(I)$. When training the image parser, we can define an equivalence set $\mathcal{G}(I,T)$ of valid trees.

%This equivalence class over trees is oblivious to the merge sequence of the background, and the image content that haven't been described in the text can also be merged in any order. Fig. \ref{} illustrates relevant situations.


%%%%%%%%%%%%%%%%%%%%%%
% \iffalse



% \fi
%%%%%%%%%%%%%%%%%%%%%%%%





\section{Weakly-supervised Model Training}

Compared with some other weak annotations such as labels and attributes, sentences usually provide richer semantics and structured contexts (\eg, object interactions and relations). More importantly, describing images by sentences finely accords with the process of human perception, and it thus contributes to meaningful representation learning.

In the initial stage of model training, we first convert each sentence into a normalized tree by using common techniques, as discussed above. Formally, a semantic tree $T$ includes entity labels (\ie, nouns) and the relations (\ie, verb phrases).

Since the scene configurations are unavailable for the training images, we need to estimate them to training our CNN and RNN. Thus, we train the model with a EM type algorithm. This algorithm alternates between predicting the latent scene configurations (via transferring knowledge from the semantic trees), and optimizing the neural network parameters.

Our model performs two tasks: semantic labeling and scene structure discovery. Thus we define the loss function as the sum of two terms: semantic label loss $\mathcal{J}_{C}$ produced by CNN, and scene structure loss $\mathcal{J}_{R}$ produced by RNN. With a training set containing $Z$ image-tree pairs $\{(I_1,T_1),...,(I_Z,T_Z)\}$. The overall loss function is as follows,
\begin{equation}\label{eq_overallloss}
\mathcal{J}(W)  =  \frac{1}{Z} \sum_{i=1}^Z ( \mathcal{J}_{C}(W_C;I_i,T_i) + \mathcal{J}_{R}(W;V_i,T_i) )
\end{equation}
where $I_i$ is the $i$-th image and $T_i$ is the tree sructure produced from the descriptive sentence. $V_i$ is the set of semantic entity features produced by CNN from the $i$-th image. $V$ takes the form $V=\{v_k|k\in \psi(T)\}$, where $\psi(T)$ is set of object categories mentioned in $T$. $W$ is all model parameters, $W_C$ is model parameters of the CNN.

% We define the intermediate label map $C$ as a latent variable. $C$ is produced by CNN normalized with softmax function. In addition the procedure in inference phase described in \ref{sub:cnn_model}, we use infromation from the descriptive text $T_i$ to further filter the label map. First, we extract the set of entity categories present in $T_i$, denoted by $\hat{E}_i$. Then for each label $l$ in $\hat{C}$, it is preserved iff $l\in E_i$, otherwise it is set to the background category. $C$ is used as estimated supervision information for training the CNN, as well as input to the RNN for the other two tasks. With EM setting, the overall loss is expressed as:
% \begin{equation}\label{eq_overallloss_em}
% \mathcal{J}(W)  =  \frac{1}{V} \sum_{i=1}^V ( \mathcal{J}_{C}(W_C;I_i;T_i) + \mathcal{J}_{R}(W_R;C_i;T_i) )
% \end{equation}
% where $C_i$ is the estimated semantic entity of image $i$.

\begin{figure}[t]
\centering
\includegraphics[width= 3.4 in]{fig_learning}
\caption{An illustration of the training process with our CNN-RNN architecture. The learning objective consists of two proportions: the semantic object labeling via the CNN, and the structure prediction via the RNN.}
\label{fig:learning}
\end{figure}

\subsection{Semantic Label Loss} \label{sub:cnn_loss}
Given intermediate label map $C$, the semantic label task performed by CNN can be optimized as a pixel-wise classfication problem. We first perform an inference step to obtain an estimated ground truth label map $\widehat{C}$, which is used as supervision (see Sect.~\ref{sub:em_learning} for more details). Let $\widehat{c}_j \in \widehat{C}$ denote the estimated category label of pixel $j$, the loss function of semantic labeling for image $I$ is defined as,


\iffalse
%\begin{equation}\label{eq_cnn_loss}
%\begin{split}
%&\mathcal{J}_{C}(W_C;I,T) = -\frac{1}{M}( \sum_{j=1}^M \sum_{k=1}^K %\textbf{1}(\widehat{c}_j=k)\log F_k(\phi_j(I,W_C))\\
%& +(1-\textbf{1}(\widehat{c}_j=k))\log(1-F_k(\phi_j(I,W_C))) ) + \|W_C\|^2
%\end{split}
%\end{equation}

\fi


\begin{equation}\label{eq_cnn_loss}
\begin{split}
&\mathcal{J}_{C}(W_C;I,T) = -\frac{1}{M}( \sum_{j=1}^M \sum_{k=1}^K \textbf{1}(\widehat{c}_j=k)\log \sigma(s_j^k)\\
& +(1-\textbf{1}(\widehat{c}_j=k))\log(1-\sigma(s_j^k)) ) + \|W_C\|^2
\end{split}
\end{equation}
where $M$ denotes the total number of pixels in the image $I$. As defined in Eq.(\ref{eq_softmax})function $\sigma(s_j^k)$ outputs the probability of $j$-th pixel for the $k$-th entity category predicted by the CNN. Note that $\{s^0,...,s^K\}$ represent the score maps of image $I$ produced by the fully convolutional network with parameters $W_C$.


\subsection{Scene Structure Loss} \label{sub:rnn_loss}
The scene structure discovery task is performed by the RNN, and can be further divided into two sub-tasks: tree structure construction and relation categorization. Thus we define the RNN loss to be the sum of loss from the two tasks,
\begin{equation}\label{eq_rnn_loss}
	\mathcal{J}_{R}(W;V_i,T_i) = \mathcal{J}_{struc}(W;V_i,T_i) + \mathcal{J}_{rel}(W;V_i,T_i)
\end{equation}

\textbf{Tree Structure Construction.}
The goal of tree structrue construction is to learn a transformation $I\rightarrow \mathcal{P}_I$ according to the tree structure $T$. We define an image parsing tree as valid if the sequence of two regions merges is consistent with the merging order in the text parsing tree. From a valid parsing tree, we extract a sequence of ``correct'' merging operations as $\mathcal{A}(V,T)=\{a_1,...,a_{P_T}\}$. $P_T$ is the total number of merging operation in the text parsing tree $T$. This implies a contraint that the nubmer of merging operation in a tree structure always equals nubmer of merging operation in the corresponding text parse tree.

We define a loss based on the merging score $q$ produced by scorer sub-network as described in Sect.~\ref{sub:rnn_model}. For convenience, we denote merging score of operation $a$ given $V$ and $T$ as $q(a)$. Intuitively, we encourage the correct merging operation $a$ to have a larger merging score than that of incorrect merging operation $\widehat{a}$. Thus we have $q(a)\geq q(\widehat{a}) + \triangle$, where $\triangle$ is a constant margin. We define the loss function for scene structrue discovery as,
\vspace{-1mm}
\begin{equation}\label{eq_structure_loss}
\begin{split}
\mathcal{J}_{struc}(W;V,T) & =  \frac{1}{P_T} \sum_{p=1}^{P_T} [ \:\: \max_{\widehat{a}_p\notin\mathcal{A}(V,T)}
q(\widehat{a}_p)\\
& - q(a_p) + \triangle \:] + \frac{\lambda}{2}||W||^2
\end{split}
\end{equation}
where $\lambda$ is the weight of regularization term. Intuitively, this loss objective function maximizes the score of correct merging operation and minimizes incorrect merging operations. To improve efficiency, we do not minimize all incorrect merging operations, but only the one with highest score.


\textbf{Relation Categorization.}
The relation categorization task can be optimized as a softmax classification problem. We define the object function of relation categorization for image $I$ as,
\begin{equation}\label{eq_relation_loss}
\begin{split}
& \mathcal{J}_{rel}(W;V,T) = -\frac{1}{|U_T|}(  \sum_{\{kl\}} \sum_{s=1}^S \textbf{1}(r_{kl}=s)\log G_s(\theta_{kl}(V,W))  \\
& +(1-\textbf{1}(r_{kl}=s))\log(1-G_s(\theta_{kl}(V,W))  ) + \|W\|^2
\end{split}
\end{equation}
$|U_T|$ denotes the number of relation appearing in the tree structre $T$. $\{kl\}$ denotes a node merged from node $k$ and $l$. $S$ is the total number of relation categories. $r_{kl}$ denotes the ground truth relations provided by tree structure $T$ between two semantic entities. $G_s(\theta_{kl}(V,W))$ is the categorizer sub-network in the RNN(see Sect.~\ref{sub:rnn_model}), which outputs the probability that node $\{kl\}$ belongs to relation category $s$.

\subsection{Learning Algorithm} \label{sub:em_learning}

The Expectation-Maximization method is adopted to optimize the loss in Eq.\eqref{eq_overallloss}. In the E-step, guided by the sentence description, we update the intermediate label maps $C$ and the latent structured configurations together with the CNN and RNN losses. In the M-step, the parameters are updated using the back-propagation algorithm. In summary, our learning algorithm can be conducted by iteratively performing the following tree steps:

% \textbf{(i) Updating intermediate label maps $\hat{C}$ and the CNN loss.} Given an image $I$, we compute the classification probability $\sigma(s^k)$ of each pixel according to Eq.\eqref{eq_softmax}, and update the intermediate label maps by considering two types of weakly-supervised information. First, inspired by the work of cardinality potentials~\cite{DBLP:conf/uai/TarlowSZAF12}\cite{DBLP:conf/icml/LiZ14}, the score of pixel $j$ belonging to the label $k$ is calculated by $f_j(k) = \sigma(s_j^k)+\delta_k$, where $\sigma(s_j^k)$ is defined in Eq.\eqref{eq_softmax} and $\delta_k$ is class-dependent biases, which is set adaptively in a manner similar to\cite{DBLP:WeaklySegmentation}. Second, with the parsing tree $T$ generated by sentence description, we use the information from $T$ to modify the label maps by only considering the objects presented in $T$ and the background category. Finally, the CNN loss is computed according to Eq.\eqref{eq_cnn_loss}.

\textbf{(i) Updating intermediate label maps $\hat{C}$ and the CNN loss.} Given image $I$ and its semantic tree $T$, we compute the classification probability of each pixel according to Eq.\eqref{eq_softmax}. Inspired by the work of cardinality potentials~\cite{DBLP:conf/uai/TarlowSZAF12}\cite{DBLP:conf/icml/LiZ14}, the score of pixel $j$ belonging to the label $k$ is calculated by $f_j(k) = \sigma(s_j^k)+\delta_k$, where $\sigma(s_j^k)$ is defined in Eq.\eqref{eq_softmax}. $\delta_k$ is entity-dependent biases, which is set adaptively according to the prescribed proportion areas of background or foreground entity classes in the image~\cite{DBLP:WeaklySegmentation}, regarding the set of entities in $T$. The final classification result of pixel $j$ is computed by $\widehat{c}_j=\arg\max_k f_j(k)$. Finally, the CNN loss is computed according to Eq.\eqref{eq_cnn_loss}.


\textbf{(ii) Updating latent scene structures and the RNN loss.} Given the label of each pixel, we group the pixels into semantic objects and obtain the object feature representations with the method described in Sect.~\ref{sub:cnn_model}. Then we use the RNN model to infer the interaction relations and hierarchical configuration of objects, and compute the RNN loss according to Eq.\eqref{eq_structure_loss} and Eq.\eqref{eq_relation_loss}.

\textbf{(iii) Updating the CNN and RNN parameters.}  Given the intermediate label maps and latent scene structure, we can compute the gradient of the overall loss in Eq.\eqref{eq_overallloss} w.r.t. the CNN and RNN parameters. With the BP algorithm, the gradients from the semantic label loss propagate backward through all layers of CNN. The gradients from the scene structure loss first propagate recursively through the layers of RNN, and then propagate through the object features to the CNN. Thus, all the parameters of our CNN-RNN model can be learned in an end-to-end manner.



%------------------------------------------------------------------------
\section{Experiment}
\label{sec:experiments}
We first introduce the implementation details and then evaluate the performance of our proposed method for semantic labeling and structured scene parsing.

%\subsection{Experimental Setup}
%\label{sub:experimental_setup}

\textbf{Datasets.}
We conduct our experiments on PASCAL VOC 2012 segmentation benchmark \cite{pascal_voc}, which contains 4,369 images from three subsets: training (1,464 images), validation (1,449 images) and test(1,456 images). PASCAL VOC 2012 dataset has 20 foreground categories and 1 background category. To suit our task, we randomly divide images in the training and validation sets into 5 groups, and asked 5 annotators to provide one description for each image in each group respectively. Since the groundtruth labeling is unavailable for test images, we did not annotate the test set. In the semi-supervised experiments, the training set is further divided into two subsets, where one is the strongly-annotated subset and the other is the PASCAL VOC 2012 training set with sentence description. Considering the Semantic Boundaries Dataset (SBD)~\cite{sbd_dataset} provides pixel-wise labels for images from PASCAL VOC 2011, we use part of the SBD to constitute the strongly-annotated subset, which includes at most 1,464 of the 10,582 training images in our experiments.

\textbf{Annotation.}
Direct annotation of the structured parsing trees for images is time-consuming, since it requires carefully designed tools and user interface. To save annotation cost, we use the natural language descriptions instead of trees. The sentence description of an image naturally provides a tree structure to indicate the major objects along with their interaction relations~\cite{DBLP:journals/ml/Elman91}. Here we use the Stanford Parser~\cite{DBLP:conf/acl/SocherBMN13} to parse sentences and produce constituency trees, which are two-way trees with each word in a sentence as a leaf node and can serve as suitable alternative of structured image tree annotation.

\textbf{Preprocessing.}
Constituency trees from the Stanford Parser~\cite{DBLP:conf/acl/SocherBMN13} still contains irrelevant words that do not describe object category or interaction relations(\eg, adjectives). Therefore, we need to convert constituency trees into semantic trees, which only contains semantic entities and scene structure (as illustrated in Fig.~\ref{fig:language}).

\begin{figure}[t]
\centering
\includegraphics[width=3.5in]{fig_language}
\caption{An illustration of the tree conversion process. The top tree is the constituency tree generated by language parser. The middle tree is the constituency tree after POS tag filtering. The bottom tree is the converted relation tree.}
\label{fig:language}
\end{figure}

The conversion process generally involves three steps. Given a constituency tree (top tree in Fig.~\ref{fig:language}), we first filter the leaf nodes by their part-of-speech, preserving only nouns as object candidates, and verbs and prepositions as relation candidates. Second, nouns are combined and converted to object categories. Annotators sometimes use different nouns for the same category (\eg ``cat'' and ``kitten''). Thus we use the lexical relation data in WordNet~\cite{wordnet} to unify the synonyms belonging to same defined category, and convert them to the corresponding object category. Annotators may mention entities that are not in any defined object categories (\eg ``grass'' in ``a sheep stands on the grass''), which will be also removed from the trees.

Third, relations should also be recognized and refined. Denote by $R$ a set of defined relations, and $T$ the triplets in the form of $(entity1, verb/prep, entity2)$. We construct a mapping $T \rightarrow R$ to recognize relation. $R$ also contains two special relation categories: ``other'' and ``background''. The ``other'' serves as a placeholder for undefined relations. The ``background'' deals with the special cases where only one entity is recognized in a tree. In this case we merge the entity with an additional ``background'' entity, and assign ``background'' relation to their parent node.

\subsection{Semantic Labeling}
\label{sub:semantic_Labeling}

In this section, we report the results for the conventional semantic labeling task which assigns semantic label to each pixel. We consider two experimental settings, \ie weakly-supervised learning and semi-supervised learning, and adopt the pixel-wise intersection-over-union(IoU) used in PASCAL VOC segmentation challenge~\cite{pascal_voc} as the performance indicator. Note that our description annotation does not cover the exact same object classes in each image as in the pixel-wise annotation, making only partial class labels are used for training. For fair comparison, we modified the training and validation images by assigning background category to the object categories not mentioned in description sentences. Due to the labels of the test set is not available, we cannot modify the test set  and thus only report the results on the modified validation set. Visualized labeling results are shown in Fig.~\ref{fig:labeling_result}.

\textbf{Weakly-supervised Learning.}
Table \ref{tbl:result_semantic_segmentation} shows the results under the setting of weakly-supervised learning. We compare our method with MIL-ILP~\cite{pinheiro2015image}, MIL-FCN~\cite{DBLP:FCnetwork}, and DeepLab~\cite{DBLP:WeaklySegmentation}, a state-of-the-art weakly-supervised method using image labels as supervision. We perform experiments with the publicly available code of DeepLab, and our own implementation of MIL-ILP and MIL-FCN. Our method obtains the IoU of 34.3\%, outperforming DeepLab~\cite{DBLP:WeaklySegmentation} by 4\%. If we fix the parameters of the RNN with random initialization, a 2.6\% drop of IoU is observed, indicating that the RNN does help in learning the CNN. % Why is the improvement insignificant? Vanishing gradients?

\begin{table}[!h]\small
\begin{center}
\begin{tabular}{|c|c|}
\hline
Method & IoU \\
\hline
MIL-ILP~\cite{pinheiro2015image} & 29.4\% \\
\hline
MIL-FCN~\cite{DBLP:FCnetwork} & 28.3\% \\
\hline
DeepLab(weakly)~\cite{DBLP:WeaklySegmentation} & 30.3\% \\
\hline
Ours(fixed-RNN) & 31.7\% \\
\hline
Ours & \textbf{34.3}\% \\
\hline

\end{tabular}
\end{center}
\caption{PASCAL 2012 val result of weakly supervised methods}
\label{tbl:result_semantic_segmentation}
\end{table}


\textbf{Semi-supervised Learning.}
In this setting, we have access to both pixel-level (strongly) annotated data and image-level (weakly) annotated data, and our method can take advantage of both types of supervision information in the training procedure. We consider two semi-supervised strategies: waterfall and fusion. For the waterfall strategy, we first perform 8,000 iterations of strongly-supervised pre-training on the CNN, followed by 16,000 iterations of weakly-supervised training on the CNN and RNN. For the fusion strategy, we use a weighted sum of strongly-supervised and weakly-supervised loss functions to train the CNN and RNN, where we use $280$ strong samples together with weak training samples, and the loss weight is set as 1:1 (strong:weak). Table \ref{tbl:semi_supervision} shows the result on the PASCAL VOC 2012 validation set. We observe that all methods benefit significantly from semi-supervised learning. The improvement of IoU compared to weakly supervised learning is 8.9\% with 280 strongly annotated samples (strong:weak = 1:5), and is 16.6\% with 1464 strongly annotated samples (strong:weak = 1:1). Our method outperforms DeepLab~\cite{DBLP:WeaklySegmentation} by 0.7\% with 280 strong samples and fusion strategy.

\begin{figure}[tbp]
\centering
\includegraphics[width=3.4in]{fig_parsingresult}
\caption{Visualized semantic labeling results. (a) The input images; (b) The groundtruth lebeling results; (c) Our proposed method (weakly-supervised); (d) Deeplab(weakly-supervised)~\cite{DBLP:WeaklySegmentation}; (e)MIL-ILP(weakly-supervised)~\cite{pinheiro2015image}}
\label{fig:labeling_result}
\end{figure}

Given the same number of strongly annotated data, the fusion strategy outperforms the waterfall strategy by 10.2\% in terms of IoU. We observe that the accuracy of pre-training step in waterfall strategy is very high (over 95\%) on the training set. This indicates that the separated pre-training with small amount of data causes the model overfitted, making pre-training contribute little to performance improvement. Nevertheless, the fusion strategy trains the model with a combined loss for better tradeoff of the two types of supervision information, and thus can exploit the strongly annotated data without suffering from overfitting.

\begin{table}[!h]\small
\begin{center}
\begin{tabular}{|c|c|c|c|c|}

\hline
Method & \# strong & \# weak & IoU \\
\hline
MIL-ILP(fusion)~\cite{pinheiro2015image} & 280 & 1464 & 39.3\% \\
\hline
MIL-FCN(fusion)~\cite{DBLP:FCnetwork} & 280 & 1464 & 38.4\%\\
\hline
DeepLab(fusion)~\cite{DBLP:WeaklySegmentation} & 280 & 1464 & 42.5\% \\
\hline
Ours(fusion) & 280 & 1464 & 43.2\% \\
\hline
Ours(fusion) & 1464 & 1464 & 50.9\% \\
\hline
Ours(waterfall) & 280 & 1464 & 33.0\% \\
\hline

\end{tabular}
\end{center}
\caption{PASCAL 2012 val result with semi-supervised learning}
\label{tbl:semi_supervision}
\vspace{-3mm}
\end{table}


\subsection{Structured Scene Parsing}
\label{sub:structure_semantic_parsing}

In this section, we evaluate the structured scene parsing performance of the proposed method, which is measured with two metrics: relation accuracy and structure accuracy. Relation accuracy is computed recursively. Denote by $T$ a tree and $P = \{T, T_1, T_2, \ldots, T_m\}$ the set of enumerated sub-tress (including $T$) of $T$. A leaf $T_i$ is considered to be correct if it is of the same object category as the one in the ground truth tree. A non-leaf $T_i$ (with two subtrees $T_l$ and $T_r$) is considered to be correct if and only if $T_l$ and $T_r$ are both correct and the relation label $r_T$ is correct. Then, the relation accuracy is calculated as $\frac{(\# of correct subtrees)}{m+1}$, and the structure accuracy is a simplification of the relation accuracy by ignoring the relation labels in the evaluation of the correctness of $T$.

Note that not all images in the PASCAL VOC 2012 validation set can be used for structure and relation accuracy, \eg the images containing only one object, and these images should not be counted in the experiments.

% \subsubsection{Network structure and initialization}
% \label{subsub:network_structure}

% Our method depends on CNN network to obtain image feature. In practice, CNN can be pre-trained, and multiple sets of network structure and parameters are publicly available. Therefore, we conduct experiments to study the impact to performance brought by different network structure and initialization. We compare the performance of our method based on two variants of the VGG-16 and VGG-19\cite{vggnet} CNN.

% \begin{table}[!h]
% \begin{center}
% \begin{tabular}{|c|c|c|c|}

% \hline
% Network & precision & recall & F-1 \\
% \hline
% VGG-16 & n/a & n/a & n/a \\
% \hline
% VGG-19 & n/a & n/a & n/a \\
% \hline

% \end{tabular}
% \end{center}
% \caption{PASCAL 2012 result with different network structure}
% \label{tbl:network_initialization}
% \end{table}

% Table \ref{tbl:network_initialization} shows the result of our method with VGG-16 and VGG-19 networks.



To get detailed understanding of our method, we study the effect of two factors, \ie joint CNN/RNN learning and end-to-end learning, and conduct experiments with the following configurations: i) Fixed the other parameters of the CNN except for the top two layers, we update all parameters of the RNN; ii) Fixed all parameters of RNN with randomly initialized values, we update all parameters of the CNN; iii) We separate the learning of CNN and RNN, \ie we first update the CNN for 16000 iterations with the fixed RNN, and then update RNN for 16000 iterations with the fixed CNN; iv) We update both CNN and RNN in the whole process with an end-to-end and joint learning manner.

\begin{table}[!ht]\small
\begin{center}
\begin{tabular}{|c|c|c|c|c|}

\hline
CNN & RNN & struct. acc & rel. acc \\
\hline
partial fixed & updated & 57.0\% & 49.0\% \\
\hline
updated & fixed(rand init) & 40.8\% & 31.4\% \\
\hline
learnt \& fixed & updated & 60.8\% & 54.2\% \\
\hline
updated & updated & 64.2\% & 62.8\% \\
\hline

\end{tabular}
\end{center}
\caption{PASCAL 2012 result with different learning strategies}
\label{tbl:result_end_to_end_learning}
\vspace{-2mm}
\end{table}


Table \ref{tbl:result_end_to_end_learning} shows the results on the PASCAL VOC 2012 validation set. Our method with end-to-end and joint learning performs best among all training settings. The training setting with fixed RNN performs much worse than one with fixed CNN, indicating that the RNN plays a more important role for structure and relation prediction. This is reasonable since structure and relation is finally obtained by RNN. Learning CNN and RNN separately performs better than learning with either fixed, but is still worse than end-to-end and joint learning.

%\subsection{Inter-task Correlation}

% \begin{figure}[t]
% \centering
% \includegraphics[width=3.5in]{fig_acc_iou}
% \caption{IoU with accuracy}
% \label{fig:acc_iou}
% \end{figure}

% \begin{figure}[!t] \centering
% \subfigure { \label{fig:nus_hamm_dist}
% \begin{minipage}[c]{0.48\linewidth}
% \centering
% \includegraphics[width=1\columnwidth]{fig_iou_acc}
% \end{minipage}%
% }%
% \hspace{-0.0in}
% \subfigure { \label{fig:nus_precision_500}
% \begin{minipage}[c]{0.48\linewidth}
% \centering
% \includegraphics[width=1\columnwidth]{fig_acc_iou}
% \end{minipage}%
% }%


%\begin{figure}[!t] \centering
%\includegraphics[width=0.8\columnwidth]{fig_acc_iou-final}
%\caption{Results of the inter-task correlation experiments. The figure shows how segmentation and structure prediction task affect each other's performance. Improving performance of one task results in improvement of the other. \textbf{The left} shows the impact of segmentation performance on relation and structure prediction. \textbf{The right} shows the impact of structure prediction performance on semantic segmentation. In practise, the segmentation performance is improved by adding more strongly annotated training data, while the performance of structure and relation prediction is improved by considering more types of relations.}
%\label{fig:inter_task_correlation}
%\vspace{-4.5mm}
%\end{figure}




%Two groups of experiments are conducted to study the inter-task correlation of the two tasks. In the first group, we report the results with three different amounts of strongly annotated data in semi-supervised learning: i) zero strongly annotated image, ii) 280 strongly annotated images, and iii) 1464 strongly annotated images.

% \begin{figure*}[!tbp] \centering
% \subfigure[] { \label{fig:acc_iou}
% \begin{minipage}[c]{0.49\linewidth}
% \centering
% \includegraphics[width=3.4in]{fig_acc_iou}
% \end{minipage}
% }
% \hspace{-0.0in}
% \subfigure[] { \label{fig:iou_acc}
% \begin{minipage}[c]{0.49\linewidth}
% \centering
% \includegraphics[width=3.4in]{fig_iou_acc}
% \end{minipage}
% }
% % \caption{IOU_ACC}
% \label{fig:inter_task_correlation}
% \end{figure*}

%In the second group, we report the results by using three configurations of the amounts of relation information in training: i) zero relation, ii) single relation and ii) all relations. In configuration i) we ignore gradients from the categorizer (see Sec.\ref{sub:rnn_model}) sub-network of the RNN. In configuration ii) we assume all relations are of the same class, and back-propagate gradients from the categorizer. In configuration iii) we use full relation data in Sec.\ref{sub:structure_semantic_parsing}.
%
%As shown in Fig.\ref{fig:inter_task_correlation}, the semantic labeling task is strongly correlated with the structured scene parsing task. For any group of experiments, the increase of relation/structure accuracy can result in a near-linear growth of semantic labeling accuracy.

\section{Conclusion}

We have introduced a structured scene parsing method based on a deep CNN-RNN architecture, and a cost-effective mode training method by transferring knowledge from image-level descriptive sentences. We have demonstrated the effectiveness of our framework by i) generating hierarchical and relation-aware configurations from the scene images and ii) achieving more favorable scene labeling results compared to other state-of-the-art weakly-supervised methods.

There are several directions in which we intend to extend this work, such as improving our system by adding a component for object attribute parsing. Deeply combining with some language processing techniques also would be a possible way.

\section*{Acknowledgment}

This work was supported in part by Special Program for Applied Research on Super Computation of the NSFC-Guangdong Joint Fund (the second phase), in part by Guangdong Natural Science Foundation under Grant S2013050014548, in part by Program of Guangzhou Zhujiang Star of Science and Technology under Grant 2013J2200067, and in part by the Fundamental Research Funds for the Central Universities.


{\small
\bibliographystyle{ieee}
\bibliography{egbib}
}

\end{document}
