\documentclass[10pt,twocolumn,letterpaper]{article}

\usepackage{cvpr}
\usepackage{times}
\usepackage{epsfig}
\usepackage{graphicx}
\usepackage{amsmath}
\usepackage{amssymb}
\usepackage{amsthm}
\usepackage{bbm}
\usepackage{subfiles}
\usepackage{subfigure}
\renewcommand{\thesubfigure}{\Alph{subfigure})}
\usepackage[lined, boxed, ruled, commentsnumbered, noend]{algorithm2e}
\usepackage{xspace}
\usepackage{xpatch}

\usepackage[utf8]{inputenc} % allow utf-8 input
\usepackage[T1]{fontenc}    % use 8-bit T1 fonts
\usepackage{url}            % simple URL typesetting
\usepackage{booktabs}       % professional-quality tables
\usepackage{amsfonts}       % blackboard math symbols
\usepackage{nicefrac}       % compact symbols for 1/2, etc.
\usepackage{microtype}      % microtypography

%\usepackage{cleveref} % section sign -- commented out; otherwise equation reference will not work.
%\usepackage[colorinlistoftodos,prependcaption,textsize=tiny]{todonotes}  % was complaining when we included this

% formatting macros%\renewcommand*{\eg}{e.g.\@\xspace}%\renewcommand*{\etc}{etc.\@\xspace}%\renewcommand*{\ie}{i.e.\@\xspace}%\newcommand*{\etal}{et al.\@\xspace}%\renewcommand*{\vs}{vs.\@\xspace}\newcommand{\oisin}[1]{{\color{blue}\bf [OISIN: #1]}}\newcommand{\yuxin}[1]{{\color{green} \bf [YUXIN: #1]}}\newcommand{\ssu}[1]{{\color{red} \bf [SHIHAN: #1]}}\newcommand{\yy}[1]{{\color{yellow} \bf [YY: #1]}}\newcommand{\pp}[1]{{\color{green} \bf [PP: #1]}}% notations\newcommand{\dataset}{{\cal D}}\newcommand{\fracpartial}[2]{\frac{\partial #1}{\partial  #2}}\newcommand{\tset}{T}% teaching set\newcommand{\diff}{\text{diff}}\newcommand{\dens}{\text{dens}}\newcommand{\dist}{\text{dist}}\newcommand{\examples}{\mathcal{X}}\newcommand{\ex}{x}\newcommand{\numexample}{m}\newcommand{\numfea}{n}\newcommand{\STRICT}{\textsf{STRICT}\xspace}\newcommand{\EXPLAIN}{\textsf{EXPLAIN}\xspace}\newcommand{\EXPLAINCROWD}{\textsf{EXPLAIN\_CROWD}\xspace}\newcommand{\RANDEXP}{\textsf{RAND\_EXP}\xspace}\newcommand{\RANDIM}{\textsf{RAND\_IM}\xspace}\newcommand{\bx}{{\mathbf{x}}}\newcommand{\hypotheses}{{\mathcal{H}}}\newcommand{\hf}[1]{h^{(#1)}}\newcommand{\hy}{h^{(0)}}\newcommand{\hstar}{h^*}\newcommand{\err}{\textrm{err}}\newcommand{\expctover}[2]{\mathbb{E}_{#1}\!\left[#2\right]}\newcommand{\labels}{\mathcal{Y}}\newcommand{\lbl}{y}\newcommand{\Lbl}{Y}\newcommand{\features}{\mathcal{F}}\newcommand{\fea}{f}\newcommand{\Fea}{F}\newcommand{\nexist}{\textsf{null}}\newcommand{\selectedts}{\mathcal{A}}\newcommand{\unit}[1]{\mathbbm{1}\{#1\}}\newcommand{\True}{\textsf{True}}\newcommand{\False}{\textsf{False}}\newcommand{\errate}{\textrm{err}}\newcommand{\tlrc}{\epsilon}\newcommand{\istar}{i^*}\newcommand{\stcomp}[1]{{#1}^\complement}\newcommand{\sgn}{\text{sign}}\newcommand{\given}{\mid}\newcommand{\defref}[1]{Definition~\ref{#1}}\newcommand{\tableref}[1]{Table~\ref{#1}}\newcommand{\figref}[1]{Fig.~\ref{#1}}\newcommand{\eqnref}[1]{\text{Eq.}~(\ref{#1})}\newcommand{\secref}[1]{\S\ref{#1}}\newcommand{\thmref}[1]{Theorem~\ref{#1}}\newcommand{\corref}[1]{Corollary~\ref{#1}}\newcommand{\propref}[1]{Proposition~\ref{#1}}\newcommand{\lemref}[1]{Lemma~\ref{#1}}\newcommand{\algref}[1]{Algorithm~\ref{#1}}\DeclareMathOperator*{\argmin}{\arg\!\min}\DeclareMathOperator*{\argmax}{\arg\!\max}%%% Local Variables:%%% mode: latex%%% TeX-master: "nips_2017"%%% End:

% Include other packages here, before hyperref.

% If you comment hyperref and then uncomment it, you should delete
% egpaper.aux before re-running latex.  (Or just hit 'q' on the first latex
% run, let it finish, and you should be clear).
\usepackage[pagebackref=true,breaklinks=true,colorlinks,bookmarks=false]{hyperref}

\cvprfinalcopy % *** Uncomment this line for the final submission

\def\cvprPaperID{1919} % *** Enter the CVPR Paper ID here
\def\httilde{\mbox{\tt\raisebox{-.5ex}{\symbol{126}}}}

% Pages are numbered in submission mode, and unnumbered in camera-ready
\ifcvprfinal\pagestyle{empty}\fi
\begin{document}

%%%%%%%%% TITLE
%\title{Interpretable Teaching of Visual Categories to Human Learners}
\title{Teaching Categories to Human Learners with Visual Explanations}

\author{Oisin Mac Aodha\hspace{20pt}Shihan Su\hspace{20pt}Yuxin Chen\hspace{20pt}Pietro Perona\hspace{20pt}Yisong Yue\\
California Institute of Technology
}


\maketitle
%\thispagestyle{empty}

%%%%%%%%% ABSTRACT
\begin{abstract}
We study the problem of computer-assisted teaching with explanations.  
Conventional approaches for machine teaching typically only provide feedback at the instance level, \eg, the category or label of the instance.  
However, it is intuitive that clear explanations from a knowledgeable teacher can significantly improve a student's ability to learn a new concept. 
To address these existing limitations, we propose a teaching framework that provides interpretable explanations as feedback and models how the learner incorporates this additional information.  
In the case of images, we show that we can automatically generate explanations that highlight the parts of the image that are responsible for the class label.
Experiments on human learners illustrate that, on average, participants achieve better test set performance on challenging categorization tasks when taught with our interpretable approach compared to existing methods. 
\end{abstract}



%%%%%%%%% BODY TEXT
\subfile{sections/intro}
\subfile{sections/related}
\subfile{sections/method}
\subfile{sections/implementation}
\subfile{sections/experiments}
\subfile{sections/conclusion}

\small{
\noindent\textbf{Acknowledgments} We would like to thank Google for their gift to the Visipedia project, AWS Research Credits, Bloomberg, Northrop Grumman, and the Swiss NSF for their Early Mobility Postdoctoral Fellowship. Thanks also to Kareem Moussa for providing the OCT dataset and to Kun ho Kim for helping generate crowd embeddings.
}

% Northrop Grumman
% and kind donations from

{\small
\vspace{10pt}
\bibliographystyle{ieee}
\bibliography{main}
}

\clearpage
\subfile{sections/supplementary}

\end{document}
